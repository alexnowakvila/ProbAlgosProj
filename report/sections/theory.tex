\section*{Theory}
\subsection{Analysis of Stage \ref{sec:stage1}} 
This section focuses on assessing the quality of the basis given
by Proto-Algorithm \ref{alg:proto-algorithm}. More precisely, we want to prove rigorous bounds on the
approximation error
$$ \|\mtx{A}-\mtx{Q}\mtx{Q}^\adj\mtx{A}\| $$
where $\|\cdot\|$ denotes either the operator norm or Frobenius norm.

We will split the argument into two parts \footnote{The authors
argue that this bipartite proof
is common in the literature of randomized linear algebra}:
\begin{enumerate}
  \item Provide a generic error bound that depends on the interaction
  between the test matrix $\mtx{\Omega}$ and the right and left
  singular values of $\mtx{A}$. \footnote{Note that we do not deal
  with randomness yet.} 
  \item Estimate the error using the distribution of the random matrix.
  We provide both expectation and probability tail bounds for the error.
\end{enumerate}

\subsubsection{(1) Error bounds via Linear Algebra}
As we aim to compute a rank-$k$ approximation of $\mtx{A}$, we appropiately
partition the exact SVD as
\begin{equation}
\label{eq:part}
\begin{array}{@{}c@{}r@{}c@{}c@{}c@{}c@{}c}
        && k & n - k && n & \\
    \mtx{A} = \mtx{U} &\left. \begin{array}{c} \\ \\ \end{array} \!\!\! \right[ &
    \begin{array}{c} \mtx{\Sigma}_1 \\ \phantom{\mtx{\Sigma}_2} \end{array} &
    \begin{array}{c} \phantom{\mtx{\Sigma}_1} \\ \mtx{\Sigma}_2 \end{array} &
    \left] \!\!\! \begin{array}{c} \\ \\ \end{array} \right. &
    \left[\begin{array}{c} \mtx{V}_{1}^{\adj} \\ \mtx{V}_{2}^{\adj}\end{array}\right]\,
    & \begin{array}{c} k \\ n - k \\ \end{array}
\end{array}
\end{equation}

Now, let $\mtx{\Omega}_i=\mtx{V}_i\mtx{\Omega}$ for $i=1,2$. Express
$\mtx{Y}=\mtx{A}\mtx{\Omega}$ as
\begin{equation*} \label{eqn:X-struct}
    \begin{array}{@{}c@{}c@{}c}
    & \ell & \\
    \mtx{Y} = \mtx{A}\mtx{\Omega} =
        \mtx{U} \left. \begin{array}{@{}c} \\ \\ \end{array} \right[ &
    \begin{array}{c} \mtx{\Sigma}_1 \mtx{\Omega}_1 \\
    \mtx{\Sigma}_2 \mtx{\Omega}_2 \end{array} &
    \left] \begin{array}{c} k \\ n - k \end{array} \right.
    \end{array}
\end{equation*}
where $\mtx{\Sigma}_1\mtx{\Omega}_1$ controls most of the action of $\mtx{Y}$,
and $\mtx{\Sigma}_2\mtx{\Omega}_2$ is a small perturbation.

The Proto-Algorithm \ref{alg:proto-algorithm}
 computes an orthogonal basis $\mtx{Q}$ of $\Ima(\mtx{Y})$. In other
words, we can express the orthogonal projection to $\Ima(\mtx{Y})$ as
$\mtx{P}_{\mtx{Y}} = \mtx{P}_{\Ima(\mtx{Y})} = \mtx{Q}\mtx{Q}^\adj$
\footnote{We simplify the notation of the orthogonal projectoin to
$\mtx{P}_{\mtx{Y}}$}.
The following Theorem \ref{thm:main-error-bd} bounds the squared
error provides a deterministic error bound to the squared error.
\begin{theorem}[Deterministic error bound] \label{thm:main-error-bd} %\\
We have that
\begin{equation}
\label{eq:main-error-bd}
\triplenorm{ (\Id - \mtx{P}_{\mtx{Y}}) \mtx{A} }^2
    \leq \triplenorm{\mtx{\Sigma}_2}^2 + 
    \smtriplenorm{\mtx{\Sigma}_2\mtx{\Omega}_2 \mtx{\Omega}_1^\psinv}^2,
\end{equation}
where $\triplenorm{\cdot}$ denotes either the spectral norm or the
Frobenius norm.
\end{theorem}

\begin{remark} \rm
 Note that $\mtx{\Sigma}_1$ does not appear in the error bound.
EXPLAIN
\end{remark}
\begin{remark} \rm
The first term is a deterministic
clean error term; we want to compute a rank-$k$
approximation so the error can not be smaller than this term.
The second term is a random term that depends on the interaction of
the right singular values of $\mtx{A}$ amplified by $\mtx{\Sigma}_2$.
\end{remark}

We would also like to be able to analyze the power scheme described in REF,
i.e, 
$\mtx{B}=(\mtx{A}\mtx{A}^\adj)\mtx{A}=\mtx{U}\mtx{\Sigma}^{2q+1}\mtx{V}^\adj$.
The rationale behind the power scheme was that the random approximation
of the $k$-dimensional gross action of $\mtx{A}$ can be improved if we amplify
$\mtx{\Sigma}_1 - \mtx{\Sigma}_2$ by power iteration. This can be easily
verified by this simple Theorem \ref{thm:power-method}.
\begin{theorem}[Power scheme] \label{thm:power-method}
%Let $m$, $n$, and $\ell$ be positive integers such that $\ell < n \leq m$.
Let $\mtx{A}$ be an $m\times n$ matrix, and let $\mtx{\Omega}$ be an $n\times \ell$
matrix. Fix a nonnegative integer $q$, form $\mtx{B} = {(\mtx{A}^\adj \mtx{A})}^q \mtx{A}$,
and compute the sample matrix $\mtx{Z} = \mtx{B\Omega}$.  Then
$$
\norm{ (\Id - \mtx{P}_{\mtx{Z}}) \mtx{A} }
    \leq \norm{ (\Id - \mtx{P}_{\mtx{Z}}) \mtx{B} }^{1/(2q+1)}.
$$
\end{theorem}
\begin{remark} \rm
Let's consider the operator norm, i.e, $\|\mtx{\Sigma}_1\|=\sigma_{k+1}$.
Then $$
\norm{ (\Id - \mtx{P}_{\mtx{Z}}) \mtx{A} }
    \leq \norm{ (\Id - \mtx{P}_{\mtx{Z}}) \mtx{B} }^{1/(2q+1)}
    \leq \left( 1 + \smnorm{}{\mtx{\Omega}_2 \mtx{\Omega}_1^\psinv}^2 \right)^{1/(4q+2)}
        \sigma_{k+1} $$
so the power scheme shrinks the suboptimality exponentially fast.
\end{remark}

Finally, we can ask what are the consequences of truncating the SVD
of $\mtx{P}_{\mtx{Z}}\mtx{A}$, i.e, compute its best rank-$k$ approximation.
\begin{theorem}[Analysis of Truncated SVD] \label{thm:truncation}
Let $\mtx{A}$ be an $m \times n$ matrix with singular values $\sigma_1 \geq \sigma_2 \geq \sigma_3 \geq \dots$,
and let $\mtx{Z}$ be an $m \times \ell$ matrix, where $\ell \geq k$.
Suppose that $\widehat{\mtx{A}}_{(k)}$ is a best rank-$k$ approximation of $\mtx{P}_{\mtx{Z}} \mtx{A}$ with respect to the spectral norm.  Then
$$
\smnorm{}{ \mtx{A} - \widehat{\mtx{A}}_{(k)} }
  \leq \sigma_{k+1} + \norm{(\Id - \mtx{P}_{\mtx{Z}}) \mtx{A}}.
$$
\end{theorem}
\begin{remark} \rm
The result of Theorem \ref{thm:truncation} is quite pessimistic, and
in practice we observe that truncating the SVD is not that damaging in
the randomized setting.
\end{remark}

\subsubsection{(2) Bounds on the gaussian setting} 
First we start by providing a bunch of results on gaussian matrices
that will are key to prove the bounds on expectation and probability tails.
%% TECHNICAL BACKGROUND
\begin{proposition}[Expected norm of a scaled Gaussian matrix] 
\label{prop:scaled-gauss}
Fix matrices $\mtx{S}, \mtx{T}$, and draw a standard Gaussian matrix $\mtx{G}$.  Then
\begin{gather}
\left( \Expect \fnormsq{ \mtx{SGT} } \right)^{1/2}
    = \fnorm{\mtx{S}} \fnorm{\mtx{T}}
    \quad\text{and}\quad
\Expect \norm{ \mtx{SGT} }
    \leq \norm{\mtx{S}} \fnorm{\mtx{T}} + \fnorm{\mtx{S}} \norm{\mtx{T}}.
    \label{eqn:avg-specnorm}
\end{gather}
\end{proposition}
%
\begin{proposition}[Expected norm of a pseudo-inverted Gaussian matrix] 
\label{prop:gauss-inv-expect}
Draw a $k \times (k + p)$ standard Gaussian matrix $\mtx{G}$ with $k \geq 2$ and $p \geq 2$.  Then
\begin{gather}
\left( \Expect \fnormsq{ \mtx{G}^\psinv } \right)^{1/2} = \sqrt{\frac{k}{p-1}}
  \quad\text{and}\quad
\Expect \norm{ \mtx{G}^\psinv } \leq \frac{\econst\sqrt{k+p}}{p}
    \label{eqn:avg-inv-specnorm}.
\end{gather}
\end{proposition}
% 
\begin{proposition}[Concentration for functions of a Gaussian matrix] 

\label{prop:gauss-tail}
Suppose that $h$ is a Lipschitz function on matrices:
$$
\abs{ h(\mtx{X}) - h(\mtx{Y}) } \leq L \fnorm{ \mtx{X} - \mtx{Y} }
\quad\text{for all $\mtx{X}, \mtx{Y}$.}
$$
Draw a standard Gaussian matrix $\mtx{G}$.  Then
$$
\Prob{ h(\mtx{G}) \geq \Expect h(\mtx{G}) + Lt } \leq \econst^{-t^2/2}.
$$
\end{proposition}
Now, we are ready to state and proof the main theorems in expectations,
and afterwards we will confirm that the error does not oscillate too
much around the mean by proving the corresponing bounds on the tails of the
distribution.

\begin{theorem}[Average Frobenius error] \label{thm:avg-frob-error-gauss}
The expected approximation error can be bounded as follows
\begin{enumerate}
  \item
$$
\Expect \fnorm{(\Id - \mtx{P}_{\mtx{Y}}) \mtx{A}}
    \leq \sigma_{k+1}\sqrt{\left( 1 + \frac{k}{p-1} \right)r(\mtx{\Sigma_2})}
$$
\item 
$$
\Expect \norm{(\Id - \mtx{P}_{\mtx{Y}}) \mtx{A}}
    \leq  \sigma_{k+1}\left(1 + \sqrt{\frac{k}{p-1}}
    +  \frac{\econst\sqrt{k+p}}{p}\sqrt{r(\mtx{\Sigma_2})}\right)
$$
 \end{enumerate}
\end{theorem}
One interesting quantity is worth examining is
$\norm{(\Id - \mtx{P}_{\mtx{Y}}) \mtx{A}}/\sigma_{k+1}$ to check
the factor that tells how far the
approximation is from the optimal rank-$k$ approximation \ref{eq:fixed-rank}.
We observe that the suboptimality term increases
essentially as $~\sim\sqrt{k/p}$ and has a term corresponding to the 
numerical rank \ref{eq:num-rank} of the singular values corresponding
to the perturbation.
\footnote{The original statement of Theorem \ref{thm:avg-frob-error-gauss}
from \cite{halko2011finding} does not explicitly write $r(\mtx{\Sigma_2})$.
However, given that we introduced the concept of numerical rank 
\ref{eq:num-rank} and its interpretation, I found interesting to highlight its
appearance in the theorem.}

\begin{proof}
H{\"o}lder's inequality and Theorem \ref{thm:main-error-bd} give 
$$
\Expect \fnorm{ (\Id - \mtx{P}_{\mtx{Y}}) \mtx{A}}
    \leq \left( \Expect \fnormsq{(\Id - \mtx{P}_{\mtx{Y}}) \mtx{A}} \right)^{1/2}
    \leq \left( \smnorm{\rm F}{\mtx{\Sigma}_2}^2 + \Expect \smnorm{\rm F}{ \mtx{\Sigma}_2
                \mtx{\Omega}_2 \mtx{\Omega}_1^\psinv }^2 \right)^{1/2}.
$$
Then, we condition on $\mtx{\Omega}_1$ and use Proposition~\ref{prop:scaled-gauss}
and first part of Proposition~\ref{prop:gauss-inv-expect}
$$
\Expect \smnorm{\rm F}{ \mtx{\Sigma}_2 \mtx{\Omega}_2 \mtx{\Omega}_1^\psinv }^2
    = \Expect \left( \Expect \left[ \smnorm{\rm F}{ \mtx{\Sigma}_2 \mtx{\Omega}_2
\mtx{\Omega}_1^\psinv }^2 \ \big\vert \ \mtx{\Omega}_1 \right] \right)
    = \Expect \left( \fnormsq{\mtx{\Sigma}_2} \smnorm{\rm F}{\mtx{\Omega}_1^\psinv}^2 \right)
    = \fnormsq{\mtx{\Sigma}_2} \cdot \Expect \smnorm{\rm F}{\mtx{\Omega}_1^\psinv}^2
    = \frac{k}{p-1} \cdot \fnormsq{\mtx{\Sigma}_2},
$$
Putting everything together
$$
\Expect \fnorm{ (\Id - \mtx{P}_{\mtx{Y}}) \mtx{A}}
    \leq \left(1 + \frac{k}{p-1}\right)^{1/2} \fnorm{ \mtx{\Sigma}_2 }.
$$
and the first part if proved. \\
The bound on the operator norm is very similar, Theorem~\ref{thm:main-error-bd}
implies that
$$
\Expect \norm{ (\Id - \mtx{P}_{\mtx{Y}}) \mtx{A} }
    \leq \Expect \left( \normsq{\mtx{\Sigma}_2}
        + \smnorm{}{\mtx{\Sigma}_2 \mtx{\Omega}_2 \mtx{\Omega}_1^\psinv }^2 \right)^{1/2}
    \leq \norm{\mtx{\Sigma}_2} + \Expect \smnorm{}{ \mtx{\Sigma}_2 \mtx{\Omega}_2 \mtx{\Omega}_1^\psinv }.
$$
Conditioning again on $\mtx{\Omega}_1$, we can bound the expectation w.r.t
 $\mtx{\Omega}_2$
$$
\Expect \smnorm{}{ \mtx{\Sigma}_2 \mtx{\Omega}_2
\mtx{\Omega}_1^\psinv }
    \leq \Expect \left( \norm{\mtx{\Sigma}_2} \smnorm{\rm F}{\mtx{\Omega}_1^\psinv}
        + \fnorm{\mtx{\Sigma}_2} \smnorm{}{\mtx{\Omega}_1^\psinv} \right)
    \leq \norm{\mtx{\Sigma}_2} \left( \Expect \smnorm{\rm F}{\mtx{\Omega}_1^\psinv}^2 \right)^{1/2}
        + \fnorm{\mtx{\Sigma}_2} \cdot \Expect \smnorm{}{\mtx{\Omega}_1^\psinv}.
$$
Finally applying Proposition~\ref{prop:gauss-inv-expect}, we get to the final
result
$$
\Expect \smnorm{}{ \mtx{\Sigma}_2 \mtx{\Omega}_2
\mtx{\Omega}_1^\psinv }
    \leq \sqrt{\frac{k}{p-1}} \norm{\mtx{\Sigma}_2}
    + \frac{\econst\sqrt{k+p}}{p} \fnorm{\mtx{\Sigma}_2}.
$$
\end{proof}

Finally, we will state the bounds on the tails that prove that the 
previously expectation bounds are representative of the random behavior.

\begin{theorem}[Deviation bounds for the Frobenius error] \label{thm:tail-frob-error-gauss}
Frame the hypotheses of Theorem~\ref{thm:avg-frob-error-gauss}.
Assume further that $p \geq 4$.  For all $u, t \geq 1$,
$$
\fnorm{ (\Id - \mtx{P}_{\mtx{Y}}) \mtx{A} }
    \leq \left( 1 + t \cdot \sqrt{12k/p} \right)
    \left( \sum\nolimits_{j > k} \sigma_j^2 \right)^{1/2}
    + ut \cdot \frac{\econst\sqrt{k+p}}{p+1} \cdot \sigma_{k+1},
$$
with failure probability at most $5 t^{-p} + 2 \econst^{-u^2/2}$.
\end{theorem}

\begin{theorem}[Deviation bounds for the spectral error] \label{thm:tail-spec-error-gauss}
Frame the hypotheses of Theorem~\ref{thm:avg-frob-error-gauss}, and assume further that $p \geq 4$.  Then
$$
\norm{ (\Id - \mtx{P}_{\mtx{Y}}) \mtx{A} }
    \leq \left( 1 + 8 \sqrt{(k + p) \cdot p \log p} \right) \sigma_{k+1}
        + 3 \sqrt{k+p} \left( \sum\nolimits_{j > k} \sigma_j^2 \right)^{1/2},
$$
with failure probability at most $6 p^{-p}$.
\end{theorem}
Similar bounds can also be proven for the power scheme 
\cite{halko2011finding} that give a high probability guarantee for 
the bound \ref{thm:power-method}.


