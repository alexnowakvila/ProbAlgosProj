%------------------------------------------------------------------------------
% Beginning of journal.tex
%------------------------------------------------------------------------------
%
% AMS-LaTeX version 2 sample file for journals, based on amsart.cls.
%
%        ***     DO NOT USE THIS FILE AS A STARTER.      ***
%        ***  USE THE JOURNAL-SPECIFIC *.TEMPLATE FILE.  ***
%
% Replace amsart by the documentclass for the target journal, e.g., tran-l.
%
\documentclass{amsart}

\newcommand{\lsp}{\vspace{3mm}}
\newcounter{remark}[section]
%\usepackage[dvips]{color}

%\usepackage[top=26mm, bottom=26mm, left=32mm, right=32mm]{geometry}



%&LaTeX

%%% This document contains macros that will be used throughout
%%% It is very neatly organized.


%%%%% Packages

%\usepackage[notcite]{showkeys}

\usepackage{amsmath}
\usepackage{amssymb}
\usepackage{bm}
\usepackage{mathrsfs}
% \usepackage[dvips]{graphicx}
%\usepackage{stmaryrd}
%\usepackage{wasysym}

\usepackage{color}

%\usepackage{url}
%\usepackage{citesort}

%\usepackage{supertabular}


%%% Algorithm package

%\usepackage{alg}

%\renewcommand{\thealgorithmfloat}{\thesection.\arabic{algorithmfloat}}

%% Commands dealing with alg.sty
\makeatletter
%\def\algspacing{\alg@unmargin}
\makeatother

\newcommand{\algfor}[1]{\textbf{for} #1\\\algbegin}


%%% Theorem setup

% \newtheorem{fact}{Fact}

%\theoremstyle{plain}

%\newtheorem{bigthm}{Theorem}
%\renewcommand{\thebigthm}{\Alph{bigthm}}    % Number with A, B, C, etc.
%\newtheorem{thm}{Theorem}
%\newtheorem{cor}[thm]{Corollary}
%\newtheorem{lemma}[thm]{Lemma}
%\newtheorem{prop}[thm]{Proposition}

%\theoremstyle{definition}
%\newtheorem{defn}[thm]{Definition}

%\newtheorem{algorithm}[thm]{Algorithm}

%\theoremstyle{remark}
%\newtheorem{remark}[theorem]{Remark}

%\newtheorem*{notation}{Notation}

%\numberwithin{equation}{section}
%\numberwithin{thm}{section}

%\numberwithin{figure}{chapter}
%\numberwithin{table}{chapter}


%%% Typesetting

\DeclareMathAlphabet{\mathsfsl}{OT1}{cmss}{m}{sl}


%%% Other font changes

\newcommand{\lang}{\textit}
\newcommand{\titl}{\textsl}
\newcommand{\term}{\emph}

%\newcommand{\algname}{\textsc}

\newcommand{\cnst}[1]{\mathrm{#1}}


%%% New environments

\newenvironment{inputs}%
    {\makebox{\phantom{}} \\ %
        \textsc{Input:}\begin{itemize}}%
    {\end{itemize}}

\newenvironment{outputs}%
    {\textsc{Output:}\begin{itemize}}%
    {\end{itemize}}

\newenvironment{procedure}%
    {\textsc{Procedure:}\begin{enumerate}}%
    {\end{enumerate}}


%%% Old symbols with new names

\newcommand{\oldphi}{\phi}
\renewcommand{\phi}{\varphi}

\newcommand{\eps}{\varepsilon}


%%% New symbols

\newcommand{\defby}{\overset{\mathrm{\scriptscriptstyle{def}}}{=}}
\newcommand{\half}{\tfrac{1}{2}}
\newcommand{\third}{\tfrac{1}{3}}



%%% Constants

\newcommand{\econst}{\mathrm{e}}
\newcommand{\iunit}{\mathrm{i}}

\newcommand{\onevct}{\mathbf{e}}
\newcommand{\zerovct}{\vct{0}}

%\newcommand{\Id}{\mathsf{I}}
%\newcommand{\onemtx}{\mathsf{1}}
%\newcommand{\zeromtx}{\mathsf{0}}

\newcommand{\Id}{\mathbf{I}}
\newcommand{\onemtx}{\bm{1}}
\newcommand{\zeromtx}{\bm{0}}


%%% Sets

\newcommand{\coll}[1]{\mathscr{#1}}

\newcommand{\Sspace}[1]{\mathbb{S}^{#1}}

\newcommand{\Rplus}{\mathbb{R}_{+}}
\newcommand{\Rspace}[1]{\mathbb{R}^{#1}}
\newcommand{\Cspace}[1]{\mathbb{C}^{#1}}

\newcommand{\RPspace}[1]{\mathbb{P}^{#1}(\mathbb{R})}
\newcommand{\CPspace}[1]{\mathbb{P}^{#1}(\mathbb{C})}
\newcommand{\FPspace}[1]{\mathbb{P}^{#1}(\mathbb{F})}

\newcommand{\RGspace}[2]{\mathbb{G}( {#1}, \Rspace{#2} )}
\newcommand{\CGspace}[2]{\mathbb{G}( {#1}, \Cspace{#2} )}
\newcommand{\FGspace}[2]{\mathbb{G}( {#1}, \mathbb{F}^{#2} )}

\newcommand{\RR}{$\Rspace{}$}
\newcommand{\CC}{$\Cspace{}$}

\newcommand{\Mset}[1]{\mathbb{M}_{#1}}

\newcommand{\oneton}[1]{\left\llbracket {#1} \right\rrbracket}


%%% Real and complex analysis

% \newcommand{\abs}[1]{\left\vert {#1} \right\vert}
\newcommand{\abssq}[1]{{\abs{#1}}^2}

\newcommand{\sgn}[1]{\operatorname{sgn}{#1}}
\newcommand{\real}{\operatorname{Re}}
\newcommand{\imag}{\operatorname{Im}}

\newcommand{\diff}[1]{\mathrm{d}{#1}}
\newcommand{\idiff}[1]{\, \diff{#1}}

%\newcommand{\grad}{\nabla}
\newcommand{\subdiff}{\partial}

\newcommand{\argmin}{\operatorname*{arg\; min}}
\newcommand{\argmax}{\operatorname*{arg\; max}}

\newcommand{\psdge}{\succcurlyeq}
\newcommand{\psdle}{\preccurlyeq}

%%% Probability

\newcommand{\Probe}[1]{\mathbb{P}\left({#1}\right)}
\newcommand{\Prob}[1]{\mathbb{P}\left\{ {#1} \right\}}
\newcommand{\Expect}{\operatorname{\mathbb{E}}}
\newcommand{\Var}{\operatorname{Var}}

\newcommand{\normal}{\textsc{normal}}
\newcommand{\erf}{\operatorname{erf}}



%%% Vector and matrix operators

\newcommand{\vct}[1]{\bm{#1}}
\newcommand{\mtx}[1]{\bm{#1}}
\DeclareMathOperator{\Ima}{Im}
%\newcommand{\mtx}[1]{\mathsf{#1}}
%\newcommand{\mtx}[1]{\mathsfsl{#1}}


\newcommand{\transp}{\mathrm{T}}
\newcommand{\adj}{*}
\newcommand{\psinv}{\dagger}

\newcommand{\lspan}[1]{\operatorname{span}{#1}}

\newcommand{\range}{\operatorname{range}}
\newcommand{\colspan}{\operatorname{colspan}}

%\newcommand{\rank}{\operatorname{rank}}
\newcommand{\strank}{\operatorname{st.rank}}

%\newcommand{\diag}{\operatorname{diag}}
\newcommand{\trace}{\operatorname{trace}}

%\newcommand{\supp}[1]{\operatorname{supp}(#1)}

\newcommand{\smax}{\sigma_{\max}}
\newcommand{\smin}{\sigma_{\min}}


%%% Mensuration: inner products and norms

\newcommand{\ip}[2]{\left\langle {#1},\ {#2} \right\rangle}
\newcommand{\absip}[2]{\abs{\ip{#1}{#2}}}
\newcommand{\abssqip}[2]{\abssq{\ip{#1}{#2}}}
\newcommand{\tworealip}[2]{2 \, \real{\ip{#1}{#2}}}

\newcommand{\norm}[1]{\left\Vert {#1} \right\Vert}
\newcommand{\normsq}[1]{\norm{#1}^2}

    % Fixed-size inner products and norms are useful sometimes

\newcommand{\smip}[2]{\bigl\langle {#1}, \ {#2} \bigr\rangle}
\newcommand{\smabsip}[2]{\bigl\vert \smip{#1}{#2} \bigr\vert}
\newcommand{\smnorm}[2]{{\bigl\Vert {#2} \bigr\Vert}_{#1}}

\newcommand{\enorm}[1]{\norm{#1}_2}
\newcommand{\enormsq}[1]{\enorm{#1}^2}

\newcommand{\fnorm}[1]{\norm{#1}_{\mathrm{F}}}
\newcommand{\fnormsq}[1]{\fnorm{#1}^2}

\newcommand{\pnorm}[2]{\norm{#2}_{#1}}
\newcommand{\infnorm}[1]{\norm{#1}_{\infty}}

\newcommand{\iinorm}[1]{\pnorm{\infty,\infty}{#1}}

\newcommand{\rxnorm}[1]{\pnorm{\mathrm{rx}}{#1}}

\newcommand{\dist}{\operatorname{dist}}

\newcommand{\triplenorm}[1]{\left\vert\!\left\vert\!\left\vert {#1} \right\vert\!\right\vert\!\right\vert}

\newcommand{\smtriplenorm}[1]{\big\vert\!\big\vert\!\big\vert {#1} \big\vert\!\big\vert\!\big\vert}

%%% Unusual operators

\newcommand{\cover}{\operatorname{cover}}
\newcommand{\pack}{\operatorname{pack}}
\newcommand{\maxcor}{\operatorname{maxcor}}
\newcommand{\ERC}{\operatorname{ERC}}
\newcommand{\quant}{\operatorname{quant}}
\newcommand{\conv}{\operatorname{conv}}
\newcommand{\spark}{\operatorname{spark}}

%%% Problem Names

\newcommand{\exactprob}{\textsc{(exact)}}
\newcommand{\rexact}{\textsc{(r-exact)}}
\newcommand{\errorprob}{\textsc{(error)}}
\newcommand{\rerror}{\textsc{(r-error)}}
\newcommand{\sparseprob}{\textsc{(sparse)}}
\newcommand{\subsetprob}{\textsc{(subset)}}
\newcommand{\rsubset}{\textsc{(r-subset)}}

\newcommand{\lonepen}{\textsc{($\ell_1$-penalty)}}
\newcommand{\loneerr}{\textsc{($\ell_1$-error)}}


%%% Things that get typed a lot

\newcommand{\Dict}{\coll{D}}

\newcommand{\opt}{\mathrm{opt}}
\newcommand{\bad}{\mathrm{bad}}
\newcommand{\err}{\mathrm{err}}
% \newcommand{\alt}{\mathrm{alt}}
\newcommand{\good}{\mathrm{good}}

\newcommand{\subjto}{\quad\text{subject to}\quad}
\newcommand{\etal}{et al.\ }

\newcommand{\cost}[2]{\operatorname{cost}_{#1}(#2)}

\newcommand{\TF}{\mathrm{TF}}

\newcommand{\chord}{\mathrm{chord}}
\newcommand{\proj}{\mathrm{spec}}
\newcommand{\fs}{\mathrm{FS}}

\newcommand{\bigO}{\mathrm{O}}


%%% Constants, vectors and matrices with names

\newcommand{\atom}{\vct{\phi}}
\newcommand{\Fee}{\mtx{\Phi}}

\newcommand{\Lamopt}{\Lambda_{\opt}}

\newcommand{\astar}{\vct{a}_{\star}}
\newcommand{\aLam}{\vct{a}_{\Lambda}}
\newcommand{\aopt}{\vct{a}_{\opt}}

\newcommand{\Astar}{\mtx{A}_{\star}}
\newcommand{\ALam}{\mtx{A}_{\Lambda}}
\newcommand{\Aopt}{\mtx{A}_{\opt}}

\newcommand{\cLam}{\vct{c}_{\Lambda}}
\newcommand{\copt}{\vct{c}_{\opt}}

\newcommand{\CLam}{\mtx{C}_{\Lambda}}
\newcommand{\Copt}{\mtx{C}_{\opt}}

\newcommand{\bstar}{\vct{b}_{\star}}
\newcommand{\bLam}{\vct{b}_{\Lambda}}
\newcommand{\bbad}{\vct{b}_{\bad}}
\newcommand{\balt}{\vct{b}_{\alt}}
\newcommand{\bopt}{\vct{c}_{\opt}}

\newcommand{\Bstar}{\mtx{B}_{\star}}
\newcommand{\BLam}{\mtx{B}_{\Lambda}}
\newcommand{\Bopt}{\mtx{B}_{\opt}}
\newcommand{\Balt}{\mtx{B}_{\alt}}

\newcommand{\Ropt}{\mtx{R}_{\opt}}


\newcommand{\gamstar}{\gamma_{\star}}

\newcommand{\PhiLam}{\mtx{\Phi}_{\Lambda}}
\newcommand{\Phiopt}{\mtx{\Phi}_{\opt}}
\newcommand{\Phialt}{\mtx{\Phi}_{\alt}}
\newcommand{\Psiopt}{\mtx{\Psi}}
\newcommand{\PsiLam}{\mtx{\Psi}_{\Lambda}}


\newcommand{\Popt}{{\mtx{P}_{\opt}}}
\newcommand{\PLam}{{\mtx{P}_{\Lambda}}}

\newcommand{\GLam}{{\mtx{G}_{\Lambda}}}

\newcommand{\rhoerr}{\rho_{\err}}
\newcommand{\rhoopt}{\rho_{\opt}}

\newcommand{\success}{\mathrm{succ}}
\newcommand{\failure}{\mathrm{fail}}

\newcommand{\restrict}[1]{\vert_{#1}}


\newcommand{\anum}[1]{{\footnotesize{#1}\quad}}

%%%%% Stupid LaTeX tricks

\newcommand{\forcetall}{\phantom{\frac{1}{\frac{1}{1}}}}

%\input{algorithm_labels.aux}
%     If your article includes graphics, uncomment this command.
\usepackage{graphicx}
\usepackage[backend=bibtex,style=numeric,natbib=true]{biblatex}
% Use the bibtex backend with the authoryear citation style (which resembles APA)
\usepackage{url}
\usepackage{citesort}
\usepackage{rotating}
\usepackage{geometry}
\usepackage{float}
\geometry{left=30mm, top=20mm}
\usepackage{hyperref}
\addbibresource{mainbib.bib} % The filename of the bibliography

\newtheorem{theorem}{Theorem}[section]
\newtheorem{lemma}[theorem]{Lemma}
\newtheorem{proposition}[theorem]{Proposition}
\newtheorem{corollary}[theorem]{Corollary}

\theoremstyle{definition}
\newtheorem{definition}[theorem]{Definition}
\newtheorem{example}[theorem]{Example}
\newtheorem{xca}[theorem]{Exercise}

\theoremstyle{remark}
\newtheorem{remark}[theorem]{Remark}
\numberwithin{equation}{section}


\newcommand*{\defeq}{\mathrel{\vcenter{\baselineskip0.5ex \lineskiplimit0pt
                     \hbox{\scriptsize.}\hbox{\scriptsize.}}}%
                     =}

\newcommand{\la}{\langle}
\newcommand{\ra}{\rangle}

%    Absolute value notation
\newcommand{\abs}[1]{\lvert#1\rvert}

%    Blank box placeholder for figures (to avoid requiring any
%    particular graphics capabilities for printing this document).
\newcommand{\blankbox}[2]{%
  \parbox{\columnwidth}{\centering
%    Set fboxsep to 0 so that the actual size of the box will match the
%    given measurements more closely.
    \setlength{\fboxsep}{0pt}%
    \fbox{\raisebox{0pt}[#2]{\hspace{#1}}}%
  }%
}

\begin{document}

\title{Probabilistic Algorithms for Finding \\ Matrix Decompositions}

%    Information for first author
\author{Alex Nowak}
%    Address of record for the research reported here
\address{Department of Mathematics \\
\'Ecole Normale Sup\'erieure de Cachan}
%    Current address
% \curraddr{Department of Mathematics and Statistics,
% Case Western Reserve University, Cleveland, Ohio 43403}
\email{alexnowakvila@gmail.com}
%    \thanks will become a 1st page footnote.
% \thanks{The first author was supported in part by NSF Grant \#000000.}

% %    Information for second author
% \author{Author Two}
% \address{Mathematical Research Section, School of Mathematical Sciences,
% Australian National University, Canberra ACT 2601, Australia}
% \email{two@maths.univ.edu.au}
% \thanks{Support information for the second author.}

%    General info
% \subjclass[2000]{Primary 54C40, 14E20; Secondary 46E25, 20C20}

% \date{January 1, 2001 and, in revised form, June 22, 2001.}

% \dedicatory{This paper is dedicated to our advisors.}

% \keywords{Differential geometry, algebraic geometry}

\begin{abstract}
Low-rank matrix approximations are obliquous in many areas ranging
from data analysis to scientific computing. From a data science point of
view, probably the most important application is due to Principal
Component Analysis (PCA), which aims to reveal hidden linear structure in
massive datasets through a low-rank matrix decomposition. 
Consequently, the complexity of the algorithm plays a central role in the
applicability of the algorithms to big data. The most common
approximative factorization
is the so-called truncated singular value decomposition (k-SVD) which can be 
computed in $\mathcal{O}(mnk)$ floating-point operations, where $k$ is the target rank
of the decomposition and $m$ and $n$ are the corresponding dimensions of the 
matrix.
In this review, we introduce to the reader randomized algorithms 
that can achieve the aforementioned task with numerous advantadges compared
to the classical algorithms. These algorithms are based on the fact that the
image of a low-rank matrix can be approximated by the action
of the matrix to a reasonable amount of random vectors from the input space.
Starting from this point, it is possible to develop
algorithms that achieve a complexity of $\mathcal{O}(mn\log{k})$ for 
dense-matrices, matches the flop count of classical
Krylov subspace methods for sparse matrices with a gain in robustness, and
for large matrices that can not be stored in memory (RAM), they
achieve a constant number of passes compared to the 
$\mathcal{O}(k)$ for classical algorithms.
\end{abstract}

\maketitle

\section*{Introduction}

Matrix factorization is listed as one of the most influential set
of techniques during the 20th century ~\cite{dongarra2000guest}, among
the Fast Fourier Transform, MCMC sampling methods and others.
As Stewart ~\cite{stewart2000decompositional} argues, the principle of the
decompositional approach aims to construct computational platforms from
which a variety of problems can be solved.

Although the decompositional approach to matrix computation remains
fundamental, nowadays in the era of big data, most of the classical algorithms
are inadequate to tackle most of the problems.

The empirical covariance matrices derived from datasets are now incredibly
big, making most of the classical approaches too expensive in terms
of computation. Moreover, it is common in information sciences to have
data which is missing or innaccurate. This gives the opportunity to sacrify
some accuracy on the algorithm to gain on computation, which classical 
algorithms are not able to do. Another important aspect is the role
of data transfer in the computational cost of a given algorithm, i.e, 
techniques that may perform fewer passes over the data may be substantially
faster in practice. 

In this review, we present, analyse and test \textit{randomized} algorithms
for matrix factorizations. This set of novel techniques addresses the issues
stated above, i.e, can trade-off computation and accuracy to an arbitrary
precision, gain on robustness, and reduce the number of passes on big datasets
when data transfer is expensive.

The purpose of approximated low-rank matrix factorization is to factorize
a given matrix $\mtx{A}\in\Rspace{m\times n}$ into a product of
two smaller matrices $\mtx{B}\in\Rspace{m\times k}$ and
$\mtx{C}\in\Rspace{k\times n}$. 

\begin{equation}
\label{eq:lowrank}
\begin{array}{ccccccccccc}
\mtx{A} &\approx& \mtx{B} & \mtx{C},\\
m\times n && m \times k & k\times n.
\end{array}
\end{equation}

The matrix $\mtx{B}\times\mtx{C}$ in \ref{eq:lowrank} is called a rank-$k$ approximation of the
matrix $\mtx{A}$.

The inner dimension $k$ is called the \textit{numerical rank}
of the matrix. This
quantity differs from the \textit{algebraic rank}, which is defined as the 
dimension of the image. The numerical rank is commonly defined as follows

\begin{equation}\label{eq:num-rank}
r(\mtx{A})\defeq\frac{\|\mtx{A}\|_F^2}{\|\mtx{A}\|^2}
= \sum_{j=1}^{\min(m,n)}\left(\frac{\sigma_j}{\sigma_1}\right)^2
\end{equation}

and it gives a better understanding of how accurate a rank-$k$ approximation
can be. Note that we always have $r(\mtx{A})\leq\text{rank}(\mtx{A})$. The notion
of numerical rank appears in ~\cite{vershynin2016high} and has been studied at the Theory Reading Group
course of the master in depth
\footnote{The comparison between both notions of rank can be better understood
through the following caracterization. 
$\text{rank}(\mtx{A})=\dim(\textbf{A}B_2^n)$ and $r(\mtx{A})=d(\mtx{A}B_2^n)$
where $B_2^n$ is the euclidean ball, hence, $\mtx{A}B_2^n$ is the ellipsoid with
the axis of magnitude the singular values $\sigma_j$'s.
Here, $\dim(\cdot)$ denotes the algebraic dimension and
$d(\cdot)$ denotes the \textit{statistical dimension}.
The statistical dimension is defined as $d(T)\defeq \frac{h(T-T)^2}{\text{diam}(T)^2}
\sim \frac{w(T)^2}{\text{diam}(T)^2}$ where $h(T)^2=\Expect\sup_{t\in T}\la g, t\ra^2$
and $w(T)=\Expect\sup_{t\in T}\la g, t\ra$ is the gaussian width.
As discussed in ~\cite{vershynin2016high}, the statistical dimension is a more stable notion
of dimension, in the same way that the numerical rank is more stable than the
algebraic rank.
}.

The task of computing a low-rank approximation to a given matrix can be split into
two computational stages. The first is to construct a low dimensional subspace that
can capture the action of the matrix. The second is to restrict the matrix to the
low dimensional subspace and then compute a standard factorization (QR, SVD, etc)
of the reduced matrix.

\begin{itemize}
  \item\textbf{Stage 1:} Compute an approximate basis for the range of the input
  matrix $\mtx{A}$. We want to find a matrix $\mtx{Q}$ with a small
  number of orthonormal columns such that

  \begin{equation}\label{eq:app-basis}
    \mtx{A}\approx \mtx{Q}\mtx{Q}^\adj\mtx{A}
  \end{equation}

  The main idea is to approximate the range of the matrix via a randomized
  method. This can be accomplised by iteratively computing the image of
  random vectors from the input space and then orthogonalizing.
  All the randomness of the algorithms will belong to this first stage.

  \item\textbf{Stage 2:} Given a matrix $\mtx{Q}$ that satisfies \ref{eq:app-basis},
  compute a standard factorization (QR, SVD, etc.) of $\mtx{A}$.  Note that
  taking $\mtx{B}=\mtx{Q}$ and $\mtx{C}=\mtx{Q}^\adj\mtx{A}$ we already have
  a low rank approximation of the matrix.
   There is no randomness at this stage and only classical linear algebra computations are
  incolved.
\end{itemize}

During the rest of the introduction, we will review some basics about matrix approximation
and we will provide to the reader the basic aspects and insights of both stages, 
which will be further studied in depth in the main body.


\subsection*{Approximating the range of a matrix via randomness}

The problem of finding the best $\epsilon$-approximation of a given matrix $\mtx{A}$
 is called the \textit{fixed-precision problem}. More concretely, we are given
 a tolerance $\epsilon$ and the goal is to find a matrix $\mtx{Q}$
 with $k=k(\epsilon)$ columns such that

\begin{equation}\label{eq:fixed-precision}
\|\mtx{A} - \mtx{Q}\mtx{Q}^\adj\mtx{A}\|\leq\epsilon
\end{equation}

The goal here is to find a $\mtx{Q}$ with the smaller number of columns
possible.

Another closely related problem is the so-called \textit{fixed-rank problem}, 
which seeks to find the best rank-$k$ approximation of the matrix.

\begin{equation}
\label{eqn:fixed-rank}
\min_{{\rm rank}(\mtx{X}) \leq j}\norm{ \mtx{A} - \mtx{X} }.
\end{equation}

The Singular Value Decomposition (SVD) is key to analyze this problem. Recall that
the SVD of a matrix $\mtx{A}$ is the following decomposition

\begin{equation}\label{eq:svd}
\mtx{A} = \mtx{U}\mtx{\Sigma}\mtx{V}^\adj
= \sum_{j=1}^{\text{rank}(\mtx{A})}\sigma_j\mtx{u}_j\mtx{v}_j^\adj
\end{equation}

where $\{\mtx{u}_k\}_k, \{\mtx{v}_k\}_k$ are orthonormal basis on the output and input
space respectively, $\sigma_1\geq\sigma_2,\ldots,\sigma_{\text{rank}(\mtx{A})}\geq 0$
are the ordered singular values.

 The SVD provides an optimal answer to the
 \textit{fixed precision problem} ~\cite{mirsky1960symmetric} through the following
 important observation
 \footnote{This is a direct consequence of the Courant-Fisher's
 \textit{min-max theorem}. This theorem offers a variational
 characterization of the singular values of a given matrix $\mtx{A}$.
 $$ \sigma_i(A) = \min_{E,\dim E=i}\max_{\mtx{x}\in S(E)}\|\mtx{A}\mtx{x}\| $$
 FER UNA MINI-DEMO}.

\begin{equation}
\label{eqn:mirsky}
\min_{{\rm rank}(\mtx{X}) \leq k}\norm{ \mtx{A} - \mtx{X} } = \sigma_{j+1}.
\end{equation}

It is straightforward to check that the optimum is attained at
 $\mtx{X}_{*} = \sum_{j=1}^k\sigma_j\mtx{u}_j\mtx{v}_j^\adj$, namely, the $k$-truncated
 SVD of the matrix $\mtx{A}$. More precisely, we have that
$\mtx{B}=\mtx{U}\mtx{\Sigma}_{[j]}^{1/2}$ and $\mtx{C} = \mtx{\Sigma}_{[j]}^{1/2}\mtx{V}^\adj$
are the best solutions \ref{eq:lowrank} when the rank is fixed.

Let's suppose now that we know the desired rank $k$ in advance. The goal is 
to find a matrix $\mtx{Q}$ with $k+p$ orthonormal columns such that

\begin{equation}\label{eq:fixed-rank-app}
\|\mtx{A}-\mtx{Q}\mtx{Q}^\adj\mtx{A}\| \approx
\min_{{\rm rank}(\mtx{X}) \leq k}\norm{ \mtx{A} - \mtx{X} }
\end{equation}

where $p$ is called the oversampling parameter.

EXPLICACIO

\subsection*{Intuition of the randomized method to find $\mtx{Q}$}

The key observation that leverage these methods is the fact that the matrix
$\mtx{Q}$ can be found by sampling, and now the reader will obtain intuition
on how this can be done.

Suppose we seek a basis for the range of a matrix $\mtx{A}$ with algebraic rank
$k$. Random elements $\mtx{y}^{(i)}$ from the range can be computed by computing the image
of random vectors $\mtx{w}^{(i)}$ of the input space. Let us repeat this
process $k$ times:

\begin{equation}\label{eq:iter-range}
\mtx{y}^{(i)} = \mtx{A}\mtx{w}^{(i)}, \hspace{0.5cm}i=1,\ldots,k
\end{equation} 

Thanks to the randomness, the set $\{\mtx{w}^{(i)}\}_{i=1}^k$ is likely to be
in general linear position and no vector will fall in $\ker\mtx{A}$
if this is a set of measure zero under the probability measure we sampled from.
Therefore, an orthogonalization procedure gives the desired orthonormal basis.

What happens if the matrix $\mtx{A}$ has not exact algebraic rank equal to $k$?
Write $\mtx{A} = \mtx{B} + \mtx{E}$ where $\mtx{B}$ is a rank-$k$ matrix containing
the information we seek and $\mtx{E}$ a small perturbation.
We want a basis that covers the range of $\mtx{B}$, however, if we repeat
procedure \ref{eq:iter-range}, the vectors will be affected by the perturbation
and the $\{\mtx{y}^{(i)}\}_{i=1}^k$ will have small components that will make
them fall outside the desired space.

To overcome this issue, the idea is to take $p$ more samples:

\begin{equation}\label{eq:iter-range}
\mtx{y}^{(i)} = \mtx{A}\mtx{w}^{(i)} = \mtx{B}\mtx{w}^{(i)}
+ \mtx{E}\mtx{w}^{(i)}, \hspace{0.5cm}i=1,\ldots,k+p
\end{equation}

The enriched set $\{\mtx{y}^{(i)}\}_{i=1}^{k+p}$ has much more chance
of spanning the desired subspace, and this is grounded with some
theoretical results holding in high probability. The theory also shows that
$p$ can be quite small. In practice, $p=5$ is more than enough.

\begin{figure}[ht]
\begin{center}
\framebox{\begin{minipage}{.9\textwidth}
\begin{center}
\textsc{Proto-Algorithm: Solving the Fixed-Rank Problem}
\end{center}
\begin{tabbing}
\hspace{5mm} \= \hspace{5mm} \= \hspace{5mm} \= \hspace{5mm} \= \kill
\anum{1} \>Draw a random $n \times (k + p)$ test matrix $\mtx{\Omega}$.\\
\anum{2} \>Form the matrix product $\mtx{Y} = \mtx{A\Omega}$.\\
\anum{3} \>Construct a matrix $\mtx{Q}$ whose columns form an orthonormal basis for \\
         \>the range of $\mtx{Y}$.
\end{tabbing}
\end{minipage}}
\end{center}
\end{figure}

\subsection*{Construction of standard matrix factorizations from $\mtx{Q}$}

This corresponds to \textbf{Stage 2} of the algorithm. Once we have $\mtx{Q}$
such that $\mtx{A}\approx\mtx{Q}\mtx{Q}^\adj\mtx{A}$, taking 
$\mtx{B}=\mtx{Q}$ and $\mtx{C}=\mtx{Q}^\adj\mtx{A}$ we produce a low rank
matrix decomposition $\mtx{A}\approx\mtx{B}\mtx{C}$.



A lot of questions need to be addressed in order to turn these methods into
a technology or \textit{off-the-shelf} algorithms, namely, 
theoretical guarantees on the accuracy of the approximate factorization and
experimentally study the gap between theory and practice, i.e, how these
methods perform in practice.

\section*{Algorithms}


\section{Fixed rank problem}

\begin{figure}[h]
\begin{center}
\framebox{\begin{minipage}{.9\textwidth}
\begin{center}
\textsc{Proto-Algorithm: Solving the Fixed-Rank Problem}
\end{center}

\lsp

\textit{Given an $m\times n$ matrix $\mtx{A}$, a target rank $k$, and an oversampling parameter $p$,
this procedure computes an $m\times (k+p)$ matrix $\mtx{Q}$ whose columns are orthonormal and whose range
approximates the range of $\mtx{A}$.}

\lsp
\begin{tabbing}
\hspace{5mm} \= \hspace{5mm} \= \hspace{5mm} \= \hspace{5mm} \= \kill

\anum{1} \>Draw a random $n \times (k + p)$ test matrix $\mtx{\Omega}$.\\

\anum{2} \>Form the matrix product $\mtx{Y} = \mtx{A\Omega}$.\\

\anum{3} \>Construct a matrix $\mtx{Q}$ whose columns form an orthonormal basis for \\
         \>the range of $\mtx{Y}$.
\end{tabbing}
\end{minipage}}
\end{center}
\end{figure}


\begin{theorem} %\label{thm:intro}
Suppose that $\mtx{A}$ is a real $m \times n$ matrix.  Select
a target rank $k \geq 2$ and an oversampling parameter $p \geq 2$,
where $k + p \leq \min\{m,n\}$.
Execute the proto-algorithm with
a standard Gaussian test matrix to obtain an
$m \times (k + p)$ matrix $\mtx{Q}$ with orthonormal columns.  Then
\begin{equation}
\label{eq:intro_err_bd}
\Expect \norm{ \mtx{A} - \mtx{QQ}^\adj \mtx{A} }
    \leq \left[ 1 + \frac{4 \sqrt{k+p}}{p-1} \cdot\sqrt{\min\{m,n\}} \right] \sigma_{k+1},
\end{equation}
where $\Expect$ denotes expectation with respect to the
random test matrix and $\sigma_{k+1}$ is the $(k+1)$th
singular value of $\mtx{A}$.
\end{theorem}

The probability that the error satisfies
\begin{equation} \label{eq:intro_err_prob}
\norm{ \mtx{A} - \mtx{QQ}^\adj \mtx{A} }
    \leq \left[ 1 + 11 \sqrt{k+p} \cdot\sqrt{\min\{m,n\}} \right] \sigma_{k+1}
\end{equation}
is at least $1 - 6 \cdot p^{-p}$ under very mild assumptions on $p$.

\section{Randomized SVD}

The Randomized SVD procedure requires %\pgnotate{Changed ''performs'' to ``requires''.}
only $2(q+1)$ passes over the matrix, so it is
efficient even for matrices stored out-of-core.
The flop count satisfies
$$
T_{\rm rand SVD} = (2q+2)\,k \, T_{\rm mult} + \bigO(k^2 (m + n)),
$$
where $T_{\rm mult}$ is the flop count of a matrix--vector multiply
with $\mtx{A}$ or $\mtx{A}^\adj$.

\begin{figure}[h]
\begin{center}
\framebox{\begin{minipage}{.9\textwidth}
\begin{center}
\textsc{Prototype for Randomized SVD}
\end{center}

\lsp

\textit{Given an $m\times n$ matrix $\mtx{A}$, a target number $k$ of singular vectors,
and an exponent $q$ (say $q=1$ or $q=2$), this procedure computes an approximate
rank-$2k$ factorization $\mtx{U\Sigma V}^\adj$, where $\mtx{U}$ and $\mtx{V}$
are orthonormal, and $\mtx{\Sigma}$ is nonnegative and diagonal.}

\lsp

{\bf Stage A:}

\begin{tabbing}
\hspace{5mm} \= \hspace{5mm} \= \hspace{5mm} \= \hspace{5mm} \= \kill
\anum{1} \>Generate an $n \times 2k$ Gaussian test matrix $\mtx{\Omega}$.\\

\anum{2} \>Form $\mtx{Y} = (\mtx{AA}^\adj)^q \mtx{A\Omega}$ by multiplying alternately with $\mtx{A}$ and $\mtx{A}^\adj$.\\

\anum{3} \>Construct a matrix $\mtx{Q}$ whose columns form an orthonormal basis for \\
         \>the range of $\mtx{Y}$.
\end{tabbing}

\lsp

{\bf Stage B:}

\begin{tabbing}
\hspace{5mm} \= \hspace{5mm} \= \hspace{5mm} \= \hspace{5mm} \= \kill
\anum{4}    \>Form $\mtx{B} = \mtx{Q}^{\adj}\mtx{A}$.\\

\anum{5}    \>Compute an SVD of the small matrix: $\mtx{B} = \widetilde{\mtx{U}}\mtx{\Sigma}\mtx{V}^{\adj}$.\\

\anum{6}    \>Set $\mtx{U} = \mtx{Q}\widetilde{\mtx{U}}$.
\end{tabbing}

\lsp

{\bf Note:} The computation of $\mtx{Y}$ in Step 2 is vulnerable to round-off errors.
When high accuracy is required, we must incorporate an orthonormalization
step between each application of $\mtx{A}$
and $\mtx{A}^{\adj}$; see Algorithm \ref{alg:subspaceiteration}.
\end{minipage}}
\end{center}
\end{figure}

\begin{theorem}
Suppose that $\mtx{A}$ is a real $m \times n$ matrix.  Select
an exponent $q$ and a target number $k$ of singular vectors,
where $2 \leq k \leq 0.5 \min\{m,n\}$.
%$ \leq k \leq 0.5 \min\{m,n\}$.
Execute the Randomized SVD algorithm to obtain a rank-$2k$
factorization $\mtx{U\Sigma V}^\adj$.  Then
\begin{equation}
\label{eq:intro_pca_bd}
\Expect \norm{ \mtx{A} - \mtx{U\Sigma V}^\adj }
    \leq \left[ 1 + 4 \sqrt{\frac{2\min\{m,n\}}{k-1}} \right]^{1/(2q+1)} \sigma_{k+1},
\end{equation}
where $\Expect$ denotes expectation with respect to the
random test matrix and $\sigma_{k+1}$ is the $(k+1)$th
singular value of $\mtx{A}$.
\end{theorem}

In practice, we can truncate the approximate SVD, retaining only the
first $k$ singular values and vectors.  Equivalently, we
replace the diagonal factor $\mtx{\Sigma}$ by the matrix
$\mtx{\Sigma}_{(k)}$ formed by zeroing out all but the
largest $k$ entries of $\mtx{\Sigma}$.  For this truncated SVD, we have the error bound
\begin{equation} \label{eqn:pca_trunc}
\Expect \norm{ \mtx{A} - \mtx{U} \mtx{\Sigma}_{(k)} \mtx{V}^\adj }
  \leq \sigma_{k+1} + \left[ 1 + 4 \sqrt{\frac{2\min\{m,n\}}{k-1}} \right]^{1/(2q+1)} \sigma_{k+1}.
\end{equation}
\section{Theory}
\subsection{Analysis of Stage 1 \ref{sec:stage1}} 
This section focuses on assessing the quality of the basis given
by Proto-Algorithm \ref{alg:proto-algorithm}. More precisely, we want to prove rigorous bounds on the
approximation error
\begin{equation}\label{eq:range-error}
\|\mtx{A}-\mtx{Q}\mtx{Q}^\adj\mtx{A}\|
\end{equation}
where $\|\cdot\|$ denotes either the operator norm or Frobenius norm.

We will split the argument into two parts \footnote{The authors
argue that this bipartite proof
is common in the literature of randomized linear algebra}:
\begin{enumerate}
  \item Provide a generic error bound that depends on the interaction
  between the test matrix $\mtx{\Omega}$ and the right and left
  singular values of $\mtx{A}$. \footnote{Note that we do not deal
  with randomness yet.} 
  \item Estimate the error using the distribution of the random matrix.
  We provide both expectation and probability tail bounds for the error.
\end{enumerate}

\subsubsection{(1) Error bounds via Linear Algebra}
As we aim to compute a rank-$k$ approximation of $\mtx{A}$, we appropiately
partition the exact SVD as
\begin{equation}
\label{eq:part}
\begin{array}{@{}c@{}r@{}c@{}c@{}c@{}c@{}c}
        && k & n - k && n & \\
    \mtx{A} = \mtx{U} &\left. \begin{array}{c} \\ \\ \end{array} \!\!\! \right[ &
    \begin{array}{c} \mtx{\Sigma}_1 \\ \phantom{\mtx{\Sigma}_2} \end{array} &
    \begin{array}{c} \phantom{\mtx{\Sigma}_1} \\ \mtx{\Sigma}_2 \end{array} &
    \left] \!\!\! \begin{array}{c} \\ \\ \end{array} \right. &
    \left[\begin{array}{c} \mtx{V}_{1}^{\adj} \\ \mtx{V}_{2}^{\adj}\end{array}\right]\,
    & \begin{array}{c} k \\ n - k \\ \end{array}
\end{array}
\end{equation}

Now, let $\mtx{\Omega}_i=\mtx{V}_i\mtx{\Omega}$ for $i=1,2$. Express
$\mtx{Y}=\mtx{A}\mtx{\Omega}$ as
\begin{equation*} \label{eqn:X-struct}
    \begin{array}{@{}c@{}c@{}c}
    & \ell & \\
    \mtx{Y} = \mtx{A}\mtx{\Omega} =
        \mtx{U} \left. \begin{array}{@{}c} \\ \\ \end{array} \right[ &
    \begin{array}{c} \mtx{\Sigma}_1 \mtx{\Omega}_1 \\
    \mtx{\Sigma}_2 \mtx{\Omega}_2 \end{array} &
    \left] \begin{array}{c} k \\ n - k \end{array} \right.
    \end{array}
\end{equation*}
where $\mtx{\Sigma}_1\mtx{\Omega}_1$ controls most of the action of $\mtx{Y}$,
and $\mtx{\Sigma}_2\mtx{\Omega}_2$ is a small perturbation.

The Proto-Algorithm \ref{alg:proto-algorithm}
 computes an orthogonal basis $\mtx{Q}$ of $\Ima(\mtx{Y})$. In other
words, we can express the orthogonal projection to $\Ima(\mtx{Y})$ as
$\mtx{P}_{\mtx{Y}} = \mtx{P}_{\Ima(\mtx{Y})} = \mtx{Q}\mtx{Q}^\adj$
\footnote{We simplify the notation of the orthogonal projectoin to
$\mtx{P}_{\mtx{Y}}$}.
The following Theorem \ref{thm:main-error-bd} bounds the squared
error provides a deterministic error bound to the squared error.
\begin{theorem}[Deterministic error bound] \label{thm:main-error-bd} %\\
We have that
\begin{equation}
\label{eq:main-error-bd}
\triplenorm{ (\Id - \mtx{P}_{\mtx{Y}}) \mtx{A} }^2
    \leq \triplenorm{\mtx{\Sigma}_2}^2 + 
    \smtriplenorm{\mtx{\Sigma}_2\mtx{\Omega}_2 \mtx{\Omega}_1^\psinv}^2,
\end{equation}
where $\triplenorm{\cdot}$ denotes either the spectral norm or the
Frobenius norm.
\end{theorem}

\begin{remark} \rm
 Note that $\mtx{\Sigma}_1$ does not appear in the error bound.
\end{remark}
\begin{remark} \rm
The first term is a deterministic
clean error term; we want to compute a rank-$k$
approximation so the error can not be smaller than this term.
The second term is a random term that depends on the interaction of
the right singular values of $\mtx{A}$ amplified by $\mtx{\Sigma}_2$.
\end{remark}

We would also like to be able to analyze the power scheme described in 
\ref{alg:randomized-power-iteration},
i.e, 
$\mtx{B}=(\mtx{A}\mtx{A}^\adj)\mtx{A}=\mtx{U}\mtx{\Sigma}^{2q+1}\mtx{V}^\adj$.
The rationale behind the power scheme was that the random approximation
of the $k$-dimensional gross action of $\mtx{A}$ can be improved if we amplify
$\mtx{\Sigma}_1 - \mtx{\Sigma}_2$ by power iteration. This can be easily
verified by the following Theorem \ref{thm:power-method}.
\begin{theorem}[Power scheme] \label{thm:power-method}
%Let $m$, $n$, and $\ell$ be positive integers such that $\ell < n \leq m$.
Let $\mtx{A}$ be an $m\times n$ matrix, and let $\mtx{\Omega}$ be an $n\times \ell$
matrix. Fix a nonnegative integer $q$, form $\mtx{B} = {(\mtx{A}\mtx{A}^\adj)}^q \mtx{A}$,
and compute the sample matrix $\mtx{Z} = \mtx{B\Omega}$.  Then
$$
\norm{ (\Id - \mtx{P}_{\mtx{Z}}) \mtx{A} }
    \leq \norm{ (\Id - \mtx{P}_{\mtx{Z}}) \mtx{B} }^{1/(2q+1)}.
$$
\end{theorem}
\begin{remark} \rm
Let's consider the operator norm, i.e, $\|\mtx{\Sigma}_1\|=\sigma_{k+1}$.
Then $$
\norm{ (\Id - \mtx{P}_{\mtx{Z}}) \mtx{A} }
    \leq \norm{ (\Id - \mtx{P}_{\mtx{Z}}) \mtx{B} }^{1/(2q+1)}
    \leq \left( 1 + \smnorm{}{\mtx{\Omega}_2 \mtx{\Omega}_1^\psinv}^2 \right)^{1/(4q+2)}
        \sigma_{k+1} $$
so the power scheme shrinks the suboptimality exponentially fast.
\end{remark}

Finally, we can ask what are the consequences of truncating the SVD
of $\mtx{P}_{\mtx{Z}}\mtx{A}$, i.e, compute its best rank-$k$ approximation.
\begin{theorem}[Analysis of Truncated SVD] \label{thm:truncation}
Let $\mtx{A}$ be an $m \times n$ matrix with singular values $\sigma_1 \geq \sigma_2 \geq \sigma_3 \geq \dots$,
and let $\mtx{Z}$ be an $m \times \ell$ matrix, where $\ell \geq k$.
Suppose that $\widehat{\mtx{A}}_{(k)}$ is a best rank-$k$ approximation of $\mtx{P}_{\mtx{Z}} \mtx{A}$ with respect to the spectral norm.  Then
$$
\smnorm{}{ \mtx{A} - \widehat{\mtx{A}}_{(k)} }
  \leq \sigma_{k+1} + \norm{(\Id - \mtx{P}_{\mtx{Z}}) \mtx{A}}.
$$
\end{theorem}
\begin{remark} \rm
The result of Theorem \ref{thm:truncation} is quite pessimistic, and
in practice we observe that truncating the SVD is not that damaging in
the randomized setting.
\end{remark}

\subsubsection{(2) Bounds on the gaussian setting} 
First we start by providing a bunch of results on gaussian matrices
that will be key to prove the bounds on expectation and probability tails.
%% TECHNICAL BACKGROUND
\begin{proposition}[Expected norm of a scaled Gaussian matrix] 
\label{prop:scaled-gauss}
Fix matrices $\mtx{S}, \mtx{T}$, and draw a standard Gaussian matrix $\mtx{G}$.  Then
\begin{gather}
\left( \Expect \fnormsq{ \mtx{SGT} } \right)^{1/2}
    = \fnorm{\mtx{S}} \fnorm{\mtx{T}}
    \quad\text{and}\quad
\Expect \norm{ \mtx{SGT} }
    \leq \norm{\mtx{S}} \fnorm{\mtx{T}} + \fnorm{\mtx{S}} \norm{\mtx{T}}.
    \label{eqn:avg-specnorm}
\end{gather}
\end{proposition}
%
\begin{proposition}[Expected norm of a pseudo-inverted Gaussian matrix] 
\label{prop:gauss-inv-expect}
Draw a $k \times (k + p)$ standard Gaussian matrix $\mtx{G}$ with $k \geq 2$ and $p \geq 2$.  Then
\begin{gather}
\left( \Expect \fnormsq{ \mtx{G}^\psinv } \right)^{1/2} = \sqrt{\frac{k}{p-1}}
  \quad\text{and}\quad
\Expect \norm{ \mtx{G}^\psinv } \leq \frac{\econst\sqrt{k+p}}{p}
    \label{eqn:avg-inv-specnorm}.
\end{gather}
\end{proposition}
% 
\begin{proposition}[Concentration for functions of a Gaussian matrix] 

\label{prop:gauss-tail}
Suppose that $h$ is a Lipschitz function on matrices:
$$
\abs{ h(\mtx{X}) - h(\mtx{Y}) } \leq L \fnorm{ \mtx{X} - \mtx{Y} }
\quad\text{for all $\mtx{X}, \mtx{Y}$.}
$$
Draw a standard Gaussian matrix $\mtx{G}$.  Then
$$
\Prob{ h(\mtx{G}) \geq \Expect h(\mtx{G}) + Lt } \leq \econst^{-t^2/2}.
$$
\end{proposition}
Now, we are ready to state and proof the main theorems in expectations,
and afterwards we will confirm that the error does not oscillate too
much around the mean by proving the corresponing bounds on the tails of the
distribution.

\begin{theorem}[Average error] \label{thm:avg-frob-error-gauss}
The expected approximation error can be bounded as follows
\begin{enumerate}
  \item
$$
\Expect \fnorm{(\Id - \mtx{P}_{\mtx{Y}}) \mtx{A}}
    \leq \sigma_{k+1}\sqrt{\left( 1 + \frac{k}{p-1} \right)r(\mtx{\Sigma_2})}
$$
\item 
$$
\Expect \norm{(\Id - \mtx{P}_{\mtx{Y}}) \mtx{A}}
    \leq  \sigma_{k+1}\left(1 + \sqrt{\frac{k}{p-1}}
    +  \frac{\econst\sqrt{k+p}}{p}\sqrt{r(\mtx{\Sigma_2})}\right)
$$
 \end{enumerate}
\end{theorem}
One interesting quantity is worth examining is
$\norm{(\Id - \mtx{P}_{\mtx{Y}}) \mtx{A}}/\sigma_{k+1}$ to check
the factor that tells how far the
approximation is from the optimal rank-$k$ approximation \ref{eq:fixed-rank}.
We observe that the suboptimality term increases
essentially as $~\sim\sqrt{k/p}$ and has a term corresponding to the 
numerical rank \ref{eq:num-rank} of the singular values corresponding
to the perturbation.
\footnote{The original statement of Theorem \ref{thm:avg-frob-error-gauss}
from \cite{halko2011finding} does not explicitly write $r(\mtx{\Sigma_2})$.
However, given that we introduced the concept of numerical rank 
\ref{eq:num-rank} and its interpretation, I found interesting to highlight its
appearance in the theorem.}
We will go through the proof for the sake of illustration.
\begin{proof}
H{\"o}lder's inequality and Theorem \ref{thm:main-error-bd} give 
$$
\Expect \fnorm{ (\Id - \mtx{P}_{\mtx{Y}}) \mtx{A}}
    \leq \left( \Expect \fnormsq{(\Id - \mtx{P}_{\mtx{Y}}) \mtx{A}} \right)^{1/2}
    \leq \left( \smnorm{\rm F}{\mtx{\Sigma}_2}^2 + \Expect \smnorm{\rm F}{ \mtx{\Sigma}_2
                \mtx{\Omega}_2 \mtx{\Omega}_1^\psinv }^2 \right)^{1/2}.
$$
Then, we condition on $\mtx{\Omega}_1$ and use Proposition~\ref{prop:scaled-gauss}
and first part of Proposition~\ref{prop:gauss-inv-expect}
$$
\Expect \smnorm{\rm F}{ \mtx{\Sigma}_2 \mtx{\Omega}_2 \mtx{\Omega}_1^\psinv }^2
    = \Expect \left( \Expect \left[ \smnorm{\rm F}{ \mtx{\Sigma}_2 \mtx{\Omega}_2
\mtx{\Omega}_1^\psinv }^2 \ \big\vert \ \mtx{\Omega}_1 \right] \right)
    = \Expect \left( \fnormsq{\mtx{\Sigma}_2} \smnorm{\rm F}{\mtx{\Omega}_1^\psinv}^2 \right)
    = \fnormsq{\mtx{\Sigma}_2} \cdot \Expect \smnorm{\rm F}{\mtx{\Omega}_1^\psinv}^2
    = \frac{k}{p-1} \cdot \fnormsq{\mtx{\Sigma}_2},
$$
Putting everything together
$$
\Expect \fnorm{ (\Id - \mtx{P}_{\mtx{Y}}) \mtx{A}}
    \leq \left(1 + \frac{k}{p-1}\right)^{1/2} \fnorm{ \mtx{\Sigma}_2 }.
$$
and the first part if proved. \\
The bound on the operator norm is very similar, Theorem~\ref{thm:main-error-bd}
implies that
$$
\Expect \norm{ (\Id - \mtx{P}_{\mtx{Y}}) \mtx{A} }
    \leq \Expect \left( \normsq{\mtx{\Sigma}_2}
        + \smnorm{}{\mtx{\Sigma}_2 \mtx{\Omega}_2 \mtx{\Omega}_1^\psinv }^2 \right)^{1/2}
    \leq \norm{\mtx{\Sigma}_2} + \Expect \smnorm{}{ \mtx{\Sigma}_2 \mtx{\Omega}_2 \mtx{\Omega}_1^\psinv }.
$$
Conditioning again on $\mtx{\Omega}_1$, we can bound the expectation w.r.t
 $\mtx{\Omega}_2$
$$
\Expect \smnorm{}{ \mtx{\Sigma}_2 \mtx{\Omega}_2
\mtx{\Omega}_1^\psinv }
    \leq \Expect \left( \norm{\mtx{\Sigma}_2} \smnorm{\rm F}{\mtx{\Omega}_1^\psinv}
        + \fnorm{\mtx{\Sigma}_2} \smnorm{}{\mtx{\Omega}_1^\psinv} \right)
    \leq \norm{\mtx{\Sigma}_2} \left( \Expect \smnorm{\rm F}{\mtx{\Omega}_1^\psinv}^2 \right)^{1/2}
        + \fnorm{\mtx{\Sigma}_2} \cdot \Expect \smnorm{}{\mtx{\Omega}_1^\psinv}.
$$
Finally applying Proposition~\ref{prop:gauss-inv-expect}, we get to the final
result
$$
\Expect \smnorm{}{ \mtx{\Sigma}_2 \mtx{\Omega}_2
\mtx{\Omega}_1^\psinv }
    \leq \sqrt{\frac{k}{p-1}} \norm{\mtx{\Sigma}_2}
    + \frac{\econst\sqrt{k+p}}{p} \fnorm{\mtx{\Sigma}_2}.
$$
\end{proof}

Finally, we will state the bounds on the tails that prove that the 
previously expectation bounds are representative of the random behavior.

\begin{theorem}[Deviation bounds for the Frobenius error] \label{thm:tail-frob-error-gauss}
Frame the hypotheses of Theorem~\ref{thm:avg-frob-error-gauss}.
Assume further that $p \geq 4$.  For all $u, t \geq 1$,
$$
\fnorm{ (\Id - \mtx{P}_{\mtx{Y}}) \mtx{A} }
    \leq \left( 1 + t \cdot \sqrt{12k/p} \right)
    \left( \sum\nolimits_{j > k} \sigma_j^2 \right)^{1/2}
    + ut \cdot \frac{\econst\sqrt{k+p}}{p+1} \cdot \sigma_{k+1},
$$
with failure probability at most $5 t^{-p} + 2 \econst^{-u^2/2}$.
\end{theorem}

\begin{theorem}[Deviation bounds for the spectral error] \label{thm:tail-spec-error-gauss}
Frame the hypotheses of Theorem~\ref{thm:avg-frob-error-gauss}, and assume further that $p \geq 4$.  Then
$$
\norm{ (\Id - \mtx{P}_{\mtx{Y}}) \mtx{A} }
    \leq \left( 1 + 8 \sqrt{(k + p) \cdot p \log p} \right) \sigma_{k+1}
        + 3 \sqrt{k+p} \left( \sum\nolimits_{j > k} \sigma_j^2 \right)^{1/2},
$$
with failure probability at most $6 p^{-p}$.
\end{theorem}
Similar bounds can also be proven for the power scheme 
\cite{halko2011finding} that give a high probability guarantee for 
the bound \ref{thm:power-method}.

Finally, let's analyze the error bounds on the 
the Power Scheme \ref{alg:randomized-power-iteration} that are directly
derived from Theorem \ref{thm:power-method} using H{\"o}lder's inequality
and the bound $\smnorm{\rm F}{\Sigma_2^{2q+1}}\leq\left(\sqrt{ \min\{m,n\} - k }\right)\sigma_{k+1}^{1/(2q+1)}$.

\begin{corollary}[Average spectral error for the power scheme]
\label{cor:power-method-spec-gauss}
Frame the hypotheses of Theorem~\ref{thm:avg-frob-error-gauss}.
Define $\mtx{B} = {(\mtx{A}\mtx{A}^\adj)}^{q} \mtx{A}$ for a
nonnegative integer $q$, and construct the sample matrix $\mtx{Z} =
\mtx{B\Omega}$.  Then
$$
\Expect \norm{(\Id - \mtx{P}_{\mtx{Z}}) \mtx{A}}
    \leq \left[ 1 + \sqrt{\frac{k}{p-1}}
    + \frac{\econst\sqrt{k+p}}{p} \cdot \sqrt{ \min\{m,n\} - k } \right]^{1/(2q+1)}
    \sigma_{k+1}.
$$
\end{corollary}

From Corollary \ref{cor:power-method-spec-gauss} we observe how as we increase
$q$, the power scheme drives the extra factor in the error to one
exponentially fast.
\newpage
\section*{Experiments}
In this section, we briefly present our experiments on some of the 
algorithms presented at the previous sections. Two sets of experiments are presented.

The first tests are on powers
of random gaussian matrices of different sizes. The goal of this set of numerics
is to check the performance of the algorithms and make sure they are working
as expected. We also analyze the sharpness of the bounds dictated by the theory.
In particular, the error estimation procedure which was motivated by 
Lemma \ref{thm:aposteriori}.

The second set of experiments is ...

\subsection{Reproducibility and general overview of the implementation}
Find repo, how to launch experiments, 
\subsection{Gaussian Matrices}
\label{sec:gaussian-matrices}
\subsection{Real Dataset}

\section*{Conclusion}
Randomization schemes for approximate matrix factorization are a powerful and 
versatile tool that \textbf{every data scientist should know about}.
These methods are more robust and scalable than classical algorithms,
and despite not achieving the same level of accuracy, the error gap can be
perfectly controlled.

Moreover, these methods rely on theoretical bounds resulting from theorems
on concentration of measure,
although we have seen in Experiment \ref{fig:exp1-1} that the bounds on the
expectation can be very loose for certain matrices. One interesting direction
of research can be the sharpening of bounds like \ref{thm:avg-frob-error-gauss}
under hypothesis on the matrix $\mtx{A}$.


There are several variants in this set of methods that
are designed for matrices with different properties. One drawback
is that sometimes they can be not that easy to verify, for e.g, the decaying
rate of the spectrum.

From a more personal point of view, I have personally appreciated both the 
theoretical and experimental section of this review. I think it is extremely 
important to have nice experimental results always grounded by theory.

\printbibliography[heading=bibintoc]
% \bibliographystyle{siam}

% \bibliographystyle{amsplain}
\end{document}

%------------------------------------------------------------------------------
% End of journal.tex
%------------------------------------------------------------------------------
\grid
