%------------------------------------------------------------------------------
% Beginning of journal.tex
%------------------------------------------------------------------------------
%
% AMS-LaTeX version 2 sample file for journals, based on amsart.cls.
%
%        ***     DO NOT USE THIS FILE AS A STARTER.      ***
%        ***  USE THE JOURNAL-SPECIFIC *.TEMPLATE FILE.  ***
%
% Replace amsart by the documentclass for the target journal, e.g., tran-l.
%
\documentclass{amsart}

\newcommand{\lsp}{\vspace{3mm}}
\newcounter{remark}[section]
%\usepackage[dvips]{color}

%\usepackage[top=26mm, bottom=26mm, left=32mm, right=32mm]{geometry}



%&LaTeX

%%% This document contains macros that will be used throughout
%%% It is very neatly organized.


%%%%% Packages

%\usepackage[notcite]{showkeys}

\usepackage{amsmath}
\usepackage{amssymb}
\usepackage{bm}
\usepackage{mathrsfs}
\usepackage[dvips]{graphicx}
%\usepackage{stmaryrd}
%\usepackage{wasysym}

\usepackage{color}

%\usepackage{url}
%\usepackage{citesort}

%\usepackage{supertabular}


%%% Algorithm package

%\usepackage{alg}

%\renewcommand{\thealgorithmfloat}{\thesection.\arabic{algorithmfloat}}

%% Commands dealing with alg.sty
\makeatletter
%\def\algspacing{\alg@unmargin}
\makeatother

\newcommand{\algfor}[1]{\textbf{for} #1\\\algbegin}


%%% Theorem setup

\newtheorem{fact}{Fact}

%\theoremstyle{plain}

%\newtheorem{bigthm}{Theorem}
%\renewcommand{\thebigthm}{\Alph{bigthm}}    % Number with A, B, C, etc.
%\newtheorem{thm}{Theorem}
%\newtheorem{cor}[thm]{Corollary}
%\newtheorem{lemma}[thm]{Lemma}
%\newtheorem{prop}[thm]{Proposition}

%\theoremstyle{definition}
%\newtheorem{defn}[thm]{Definition}

%\newtheorem{algorithm}[thm]{Algorithm}

%\theoremstyle{remark}
%\newtheorem{remark}[theorem]{Remark}

%\newtheorem*{notation}{Notation}

%\numberwithin{equation}{section}
%\numberwithin{thm}{section}

%\numberwithin{figure}{chapter}
%\numberwithin{table}{chapter}


%%% Typesetting

\DeclareMathAlphabet{\mathsfsl}{OT1}{cmss}{m}{sl}


%%% Other font changes

\newcommand{\lang}{\textit}
\newcommand{\titl}{\textsl}
\newcommand{\term}{\emph}

%\newcommand{\algname}{\textsc}

\newcommand{\cnst}[1]{\mathrm{#1}}


%%% New environments

\newenvironment{inputs}%
    {\makebox{\phantom{}} \\ %
        \textsc{Input:}\begin{itemize}}%
    {\end{itemize}}

\newenvironment{outputs}%
    {\textsc{Output:}\begin{itemize}}%
    {\end{itemize}}

\newenvironment{procedure}%
    {\textsc{Procedure:}\begin{enumerate}}%
    {\end{enumerate}}


%%% Old symbols with new names

\newcommand{\oldphi}{\phi}
\renewcommand{\phi}{\varphi}

\newcommand{\eps}{\varepsilon}


%%% New symbols

\newcommand{\defby}{\overset{\mathrm{\scriptscriptstyle{def}}}{=}}
\newcommand{\half}{\tfrac{1}{2}}
\newcommand{\third}{\tfrac{1}{3}}



%%% Constants

\newcommand{\econst}{\mathrm{e}}
\newcommand{\iunit}{\mathrm{i}}

\newcommand{\onevct}{\mathbf{e}}
\newcommand{\zerovct}{\vct{0}}

%\newcommand{\Id}{\mathsf{I}}
%\newcommand{\onemtx}{\mathsf{1}}
%\newcommand{\zeromtx}{\mathsf{0}}

\newcommand{\Id}{\mathbf{I}}
\newcommand{\onemtx}{\bm{1}}
\newcommand{\zeromtx}{\bm{0}}


%%% Sets

\newcommand{\coll}[1]{\mathscr{#1}}

\newcommand{\Sspace}[1]{\mathbb{S}^{#1}}

\newcommand{\Rplus}{\mathbb{R}_{+}}
\newcommand{\Rspace}[1]{\mathbb{R}^{#1}}
\newcommand{\Cspace}[1]{\mathbb{C}^{#1}}

\newcommand{\RPspace}[1]{\mathbb{P}^{#1}(\mathbb{R})}
\newcommand{\CPspace}[1]{\mathbb{P}^{#1}(\mathbb{C})}
\newcommand{\FPspace}[1]{\mathbb{P}^{#1}(\mathbb{F})}

\newcommand{\RGspace}[2]{\mathbb{G}( {#1}, \Rspace{#2} )}
\newcommand{\CGspace}[2]{\mathbb{G}( {#1}, \Cspace{#2} )}
\newcommand{\FGspace}[2]{\mathbb{G}( {#1}, \mathbb{F}^{#2} )}

\newcommand{\RR}{$\Rspace{}$}
\newcommand{\CC}{$\Cspace{}$}

\newcommand{\Mset}[1]{\mathbb{M}_{#1}}

\newcommand{\oneton}[1]{\left\llbracket {#1} \right\rrbracket}


%%% Real and complex analysis

\newcommand{\abs}[1]{\left\vert {#1} \right\vert}
\newcommand{\abssq}[1]{{\abs{#1}}^2}

\newcommand{\sgn}[1]{\operatorname{sgn}{#1}}
\newcommand{\real}{\operatorname{Re}}
\newcommand{\imag}{\operatorname{Im}}

\newcommand{\diff}[1]{\mathrm{d}{#1}}
\newcommand{\idiff}[1]{\, \diff{#1}}

%\newcommand{\grad}{\nabla}
\newcommand{\subdiff}{\partial}

\newcommand{\argmin}{\operatorname*{arg\; min}}
\newcommand{\argmax}{\operatorname*{arg\; max}}

\newcommand{\psdge}{\succcurlyeq}
\newcommand{\psdle}{\preccurlyeq}

%%% Probability

\newcommand{\Probe}[1]{\mathbb{P}\left({#1}\right)}
\newcommand{\Prob}[1]{\mathbb{P}\left\{ {#1} \right\}}
\newcommand{\Expect}{\operatorname{\mathbb{E}}}
\newcommand{\Var}{\operatorname{Var}}

\newcommand{\normal}{\textsc{normal}}
\newcommand{\erf}{\operatorname{erf}}



%%% Vector and matrix operators

\newcommand{\vct}[1]{\bm{#1}}
\newcommand{\mtx}[1]{\bm{#1}}
%\newcommand{\mtx}[1]{\mathsf{#1}}
%\newcommand{\mtx}[1]{\mathsfsl{#1}}


\newcommand{\transp}{\mathrm{T}}
\newcommand{\adj}{*}
\newcommand{\psinv}{\dagger}

\newcommand{\lspan}[1]{\operatorname{span}{#1}}

\newcommand{\range}{\operatorname{range}}
\newcommand{\colspan}{\operatorname{colspan}}

%\newcommand{\rank}{\operatorname{rank}}
\newcommand{\strank}{\operatorname{st.rank}}

%\newcommand{\diag}{\operatorname{diag}}
\newcommand{\trace}{\operatorname{trace}}

%\newcommand{\supp}[1]{\operatorname{supp}(#1)}

\newcommand{\smax}{\sigma_{\max}}
\newcommand{\smin}{\sigma_{\min}}


%%% Mensuration: inner products and norms

\newcommand{\ip}[2]{\left\langle {#1},\ {#2} \right\rangle}
\newcommand{\absip}[2]{\abs{\ip{#1}{#2}}}
\newcommand{\abssqip}[2]{\abssq{\ip{#1}{#2}}}
\newcommand{\tworealip}[2]{2 \, \real{\ip{#1}{#2}}}

\newcommand{\norm}[1]{\left\Vert {#1} \right\Vert}
\newcommand{\normsq}[1]{\norm{#1}^2}

    % Fixed-size inner products and norms are useful sometimes

\newcommand{\smip}[2]{\bigl\langle {#1}, \ {#2} \bigr\rangle}
\newcommand{\smabsip}[2]{\bigl\vert \smip{#1}{#2} \bigr\vert}
\newcommand{\smnorm}[2]{{\bigl\Vert {#2} \bigr\Vert}_{#1}}

\newcommand{\enorm}[1]{\norm{#1}_2}
\newcommand{\enormsq}[1]{\enorm{#1}^2}

\newcommand{\fnorm}[1]{\norm{#1}_{\mathrm{F}}}
\newcommand{\fnormsq}[1]{\fnorm{#1}^2}

\newcommand{\pnorm}[2]{\norm{#2}_{#1}}
\newcommand{\infnorm}[1]{\norm{#1}_{\infty}}

\newcommand{\iinorm}[1]{\pnorm{\infty,\infty}{#1}}

\newcommand{\rxnorm}[1]{\pnorm{\mathrm{rx}}{#1}}

\newcommand{\dist}{\operatorname{dist}}

\newcommand{\triplenorm}[1]{\left\vert\!\left\vert\!\left\vert {#1} \right\vert\!\right\vert\!\right\vert}

\newcommand{\smtriplenorm}[1]{\big\vert\!\big\vert\!\big\vert {#1} \big\vert\!\big\vert\!\big\vert}

%%% Unusual operators

\newcommand{\cover}{\operatorname{cover}}
\newcommand{\pack}{\operatorname{pack}}
\newcommand{\maxcor}{\operatorname{maxcor}}
\newcommand{\ERC}{\operatorname{ERC}}
\newcommand{\quant}{\operatorname{quant}}
\newcommand{\conv}{\operatorname{conv}}
\newcommand{\spark}{\operatorname{spark}}

%%% Problem Names

\newcommand{\exactprob}{\textsc{(exact)}}
\newcommand{\rexact}{\textsc{(r-exact)}}
\newcommand{\errorprob}{\textsc{(error)}}
\newcommand{\rerror}{\textsc{(r-error)}}
\newcommand{\sparseprob}{\textsc{(sparse)}}
\newcommand{\subsetprob}{\textsc{(subset)}}
\newcommand{\rsubset}{\textsc{(r-subset)}}

\newcommand{\lonepen}{\textsc{($\ell_1$-penalty)}}
\newcommand{\loneerr}{\textsc{($\ell_1$-error)}}


%%% Things that get typed a lot

\newcommand{\Dict}{\coll{D}}

\newcommand{\opt}{\mathrm{opt}}
\newcommand{\bad}{\mathrm{bad}}
\newcommand{\err}{\mathrm{err}}
\newcommand{\alt}{\mathrm{alt}}
\newcommand{\good}{\mathrm{good}}

\newcommand{\subjto}{\quad\text{subject to}\quad}
\newcommand{\etal}{et al.\ }

\newcommand{\cost}[2]{\operatorname{cost}_{#1}(#2)}

\newcommand{\TF}{\mathrm{TF}}

\newcommand{\chord}{\mathrm{chord}}
\newcommand{\proj}{\mathrm{spec}}
\newcommand{\fs}{\mathrm{FS}}

\newcommand{\bigO}{\mathrm{O}}


%%% Constants, vectors and matrices with names

\newcommand{\atom}{\vct{\phi}}
\newcommand{\Fee}{\mtx{\Phi}}

\newcommand{\Lamopt}{\Lambda_{\opt}}

\newcommand{\astar}{\vct{a}_{\star}}
\newcommand{\aLam}{\vct{a}_{\Lambda}}
\newcommand{\aopt}{\vct{a}_{\opt}}

\newcommand{\Astar}{\mtx{A}_{\star}}
\newcommand{\ALam}{\mtx{A}_{\Lambda}}
\newcommand{\Aopt}{\mtx{A}_{\opt}}

\newcommand{\cLam}{\vct{c}_{\Lambda}}
\newcommand{\copt}{\vct{c}_{\opt}}

\newcommand{\CLam}{\mtx{C}_{\Lambda}}
\newcommand{\Copt}{\mtx{C}_{\opt}}

\newcommand{\bstar}{\vct{b}_{\star}}
\newcommand{\bLam}{\vct{b}_{\Lambda}}
\newcommand{\bbad}{\vct{b}_{\bad}}
\newcommand{\balt}{\vct{b}_{\alt}}
\newcommand{\bopt}{\vct{c}_{\opt}}

\newcommand{\Bstar}{\mtx{B}_{\star}}
\newcommand{\BLam}{\mtx{B}_{\Lambda}}
\newcommand{\Bopt}{\mtx{B}_{\opt}}
\newcommand{\Balt}{\mtx{B}_{\alt}}

\newcommand{\Ropt}{\mtx{R}_{\opt}}


\newcommand{\gamstar}{\gamma_{\star}}

\newcommand{\PhiLam}{\mtx{\Phi}_{\Lambda}}
\newcommand{\Phiopt}{\mtx{\Phi}_{\opt}}
\newcommand{\Phialt}{\mtx{\Phi}_{\alt}}
\newcommand{\Psiopt}{\mtx{\Psi}}
\newcommand{\PsiLam}{\mtx{\Psi}_{\Lambda}}


\newcommand{\Popt}{{\mtx{P}_{\opt}}}
\newcommand{\PLam}{{\mtx{P}_{\Lambda}}}

\newcommand{\GLam}{{\mtx{G}_{\Lambda}}}

\newcommand{\rhoerr}{\rho_{\err}}
\newcommand{\rhoopt}{\rho_{\opt}}

\newcommand{\success}{\mathrm{succ}}
\newcommand{\failure}{\mathrm{fail}}

\newcommand{\restrict}[1]{\vert_{#1}}


\newcommand{\anum}[1]{{\footnotesize{#1}\quad}}

%%%%% Stupid LaTeX tricks

\newcommand{\forcetall}{\phantom{\frac{1}{\frac{1}{1}}}}

%\input{algorithm_labels.aux}
%     If your article includes graphics, uncomment this command.
\usepackage{graphicx}

\newtheorem{theorem}{Theorem}[section]
\newtheorem{lemma}[theorem]{Lemma}

\theoremstyle{definition}
\newtheorem{definition}[theorem]{Definition}
\newtheorem{example}[theorem]{Example}
\newtheorem{xca}[theorem]{Exercise}

\theoremstyle{remark}
\newtheorem{remark}[theorem]{Remark}
\numberwithin{equation}{section}

%    Absolute value notation
\newcommand{\abs}[1]{\lvert#1\rvert}

%    Blank box placeholder for figures (to avoid requiring any
%    particular graphics capabilities for printing this document).
\newcommand{\blankbox}[2]{%
  \parbox{\columnwidth}{\centering
%    Set fboxsep to 0 so that the actual size of the box will match the
%    given measurements more closely.
    \setlength{\fboxsep}{0pt}%
    \fbox{\raisebox{0pt}[#2]{\hspace{#1}}}%
  }%
}

\begin{document}

\title{Probabilistic Algorithms for Finding Matrix Decompositions}

%    Information for first author
\author{Alex Nowak}
%    Address of record for the research reported here
% \address{Department of Mathematics, Louisiana State University, Baton
% Rouge, Louisiana 70803}
%    Current address
% \curraddr{Department of Mathematics and Statistics,
% Case Western Reserve University, Cleveland, Ohio 43403}
\email{alexnowakvila@gmail.com}
%    \thanks will become a 1st page footnote.
% \thanks{The first author was supported in part by NSF Grant \#000000.}

% %    Information for second author
% \author{Author Two}
% \address{Mathematical Research Section, School of Mathematical Sciences,
% Australian National University, Canberra ACT 2601, Australia}
% \email{two@maths.univ.edu.au}
% \thanks{Support information for the second author.}

%    General info
% \subjclass[2000]{Primary 54C40, 14E20; Secondary 46E25, 20C20}

% \date{January 1, 2001 and, in revised form, June 22, 2001.}

% \dedicatory{This paper is dedicated to our advisors.}

% \keywords{Differential geometry, algebraic geometry}

\begin{abstract}
This paper is a sample prepared to illustrate the use of the American
Mathematical Society's \LaTeX{} document class \texttt{amsart} and
publication-specific variants of that class for AMS-\LaTeX{} version 2.
\end{abstract}

\maketitle

\section*{Theory}

\section{Fixed rank problem}

\begin{figure}[h]
\begin{center}
\framebox{\begin{minipage}{.9\textwidth}
\begin{center}
\textsc{Proto-Algorithm: Solving the Fixed-Rank Problem}
\end{center}

\lsp

\textit{Given an $m\times n$ matrix $\mtx{A}$, a target rank $k$, and an oversampling parameter $p$,
this procedure computes an $m\times (k+p)$ matrix $\mtx{Q}$ whose columns are orthonormal and whose range
approximates the range of $\mtx{A}$.}

\lsp
\begin{tabbing}
\hspace{5mm} \= \hspace{5mm} \= \hspace{5mm} \= \hspace{5mm} \= \kill

\anum{1} \>Draw a random $n \times (k + p)$ test matrix $\mtx{\Omega}$.\\

\anum{2} \>Form the matrix product $\mtx{Y} = \mtx{A\Omega}$.\\

\anum{3} \>Construct a matrix $\mtx{Q}$ whose columns form an orthonormal basis for \\
         \>the range of $\mtx{Y}$.
\end{tabbing}
\end{minipage}}
\end{center}
\end{figure}


\begin{theorem} %\label{thm:intro}
Suppose that $\mtx{A}$ is a real $m \times n$ matrix.  Select
a target rank $k \geq 2$ and an oversampling parameter $p \geq 2$,
where $k + p \leq \min\{m,n\}$.
Execute the proto-algorithm with
a standard Gaussian test matrix to obtain an
$m \times (k + p)$ matrix $\mtx{Q}$ with orthonormal columns.  Then
\begin{equation}
\label{eq:intro_err_bd}
\Expect \norm{ \mtx{A} - \mtx{QQ}^\adj \mtx{A} }
    \leq \left[ 1 + \frac{4 \sqrt{k+p}}{p-1} \cdot\sqrt{\min\{m,n\}} \right] \sigma_{k+1},
\end{equation}
where $\Expect$ denotes expectation with respect to the
random test matrix and $\sigma_{k+1}$ is the $(k+1)$th
singular value of $\mtx{A}$.
\end{theorem}

The probability that the error satisfies
\begin{equation} \label{eq:intro_err_prob}
\norm{ \mtx{A} - \mtx{QQ}^\adj \mtx{A} }
    \leq \left[ 1 + 11 \sqrt{k+p} \cdot\sqrt{\min\{m,n\}} \right] \sigma_{k+1}
\end{equation}
is at least $1 - 6 \cdot p^{-p}$ under very mild assumptions on $p$.

\section{Randomized SVD}

The Randomized SVD procedure requires %\pgnotate{Changed ''performs'' to ``requires''.}
only $2(q+1)$ passes over the matrix, so it is
efficient even for matrices stored out-of-core.
The flop count satisfies
$$
T_{\rm rand SVD} = (2q+2)\,k \, T_{\rm mult} + \bigO(k^2 (m + n)),
$$
where $T_{\rm mult}$ is the flop count of a matrix--vector multiply
with $\mtx{A}$ or $\mtx{A}^\adj$.

\begin{figure}[h]
\begin{center}
\framebox{\begin{minipage}{.9\textwidth}
\begin{center}
\textsc{Prototype for Randomized SVD}
\end{center}

\lsp

\textit{Given an $m\times n$ matrix $\mtx{A}$, a target number $k$ of singular vectors,
and an exponent $q$ (say $q=1$ or $q=2$), this procedure computes an approximate
rank-$2k$ factorization $\mtx{U\Sigma V}^\adj$, where $\mtx{U}$ and $\mtx{V}$
are orthonormal, and $\mtx{\Sigma}$ is nonnegative and diagonal.}

\lsp

{\bf Stage A:}

\begin{tabbing}
\hspace{5mm} \= \hspace{5mm} \= \hspace{5mm} \= \hspace{5mm} \= \kill
\anum{1} \>Generate an $n \times 2k$ Gaussian test matrix $\mtx{\Omega}$.\\

\anum{2} \>Form $\mtx{Y} = (\mtx{AA}^\adj)^q \mtx{A\Omega}$ by multiplying alternately with $\mtx{A}$ and $\mtx{A}^\adj$.\\

\anum{3} \>Construct a matrix $\mtx{Q}$ whose columns form an orthonormal basis for \\
         \>the range of $\mtx{Y}$.
\end{tabbing}

\lsp

{\bf Stage B:}

\begin{tabbing}
\hspace{5mm} \= \hspace{5mm} \= \hspace{5mm} \= \hspace{5mm} \= \kill
\anum{4}    \>Form $\mtx{B} = \mtx{Q}^{\adj}\mtx{A}$.\\

\anum{5}    \>Compute an SVD of the small matrix: $\mtx{B} = \widetilde{\mtx{U}}\mtx{\Sigma}\mtx{V}^{\adj}$.\\

\anum{6}    \>Set $\mtx{U} = \mtx{Q}\widetilde{\mtx{U}}$.
\end{tabbing}

\lsp

{\bf Note:} The computation of $\mtx{Y}$ in Step 2 is vulnerable to round-off errors.
When high accuracy is required, we must incorporate an orthonormalization
step between each application of $\mtx{A}$
and $\mtx{A}^{\adj}$; see Algorithm \ref{alg:subspaceiteration}.
\end{minipage}}
\end{center}
\end{figure}

\begin{theorem}
Suppose that $\mtx{A}$ is a real $m \times n$ matrix.  Select
an exponent $q$ and a target number $k$ of singular vectors,
where $2 \leq k \leq 0.5 \min\{m,n\}$.
%$ \leq k \leq 0.5 \min\{m,n\}$.
Execute the Randomized SVD algorithm to obtain a rank-$2k$
factorization $\mtx{U\Sigma V}^\adj$.  Then
\begin{equation}
\label{eq:intro_pca_bd}
\Expect \norm{ \mtx{A} - \mtx{U\Sigma V}^\adj }
    \leq \left[ 1 + 4 \sqrt{\frac{2\min\{m,n\}}{k-1}} \right]^{1/(2q+1)} \sigma_{k+1},
\end{equation}
where $\Expect$ denotes expectation with respect to the
random test matrix and $\sigma_{k+1}$ is the $(k+1)$th
singular value of $\mtx{A}$.
\end{theorem}

In practice, we can truncate the approximate SVD, retaining only the
first $k$ singular values and vectors.  Equivalently, we
replace the diagonal factor $\mtx{\Sigma}$ by the matrix
$\mtx{\Sigma}_{(k)}$ formed by zeroing out all but the
largest $k$ entries of $\mtx{\Sigma}$.  For this truncated SVD, we have the error bound
\begin{equation} \label{eqn:pca_trunc}
\Expect \norm{ \mtx{A} - \mtx{U} \mtx{\Sigma}_{(k)} \mtx{V}^\adj }
  \leq \sigma_{k+1} + \left[ 1 + 4 \sqrt{\frac{2\min\{m,n\}}{k-1}} \right]^{1/(2q+1)} \sigma_{k+1}.
\end{equation}


\bibliographystyle{amsplain}
\begin{thebibliography}{10}

\bibitem {A} T. Aoki, \textit{Calcul exponentiel des op\'erateurs
microdifferentiels d'ordre infini.} I, Ann. Inst. Fourier (Grenoble)
\textbf{33} (1983), 227--250.

\bibitem {B} R. Brown, \textit{On a conjecture of Dirichlet},
Amer. Math. Soc., Providence, RI, 1993.

\bibitem {D} R. A. DeVore, \textit{Approximation of functions},
Proc. Sympos. Appl. Math., vol. 36,
Amer. Math. Soc., Providence, RI, 1986, pp. 34--56.

\end{thebibliography}

\end{document}

%------------------------------------------------------------------------------
% End of journal.tex
%------------------------------------------------------------------------------
