\section*{Algorithms}

In this section we will describe the randomized algorithms in detail, provide the
corresponding computational complexity analysis, and state the main theoretical results
that guarantee the accuracy of the approximation.

\section{Stage 1}
In the introductory section, we provided some intuition on the randomized procedure
and we developed a general proto-algorithm \ref{alg:basic} to find
the matrix $\mtx{Q}$. However, Algorithm \ref{alg:basic} is very general and can be
tunned depending on the problem requirements.
The number $T_{\rm basic}$ of flops required by Algorithm \ref{alg:basic} satisfies
\begin{equation}
\label{eq:cost_basic}
T_{\rm basic} \sim \ell n \, T_{\rm rand} + \ell\,T_{\rm mult} + \ell^{2}m %\,T_{\rm flop},
\end{equation}
where $T_{\rm rand}$ is the cost of generating a Gaussian random number
and $T_{\rm mult}$ is the cost of multiplying $\mtx{A}$ by a vector.

We will now describe some specific realizations of Algorithm \ref{alg:basic} 
that will be intended for problems with different requirements.

%%%%%%%%%%%%%%%%%% RANDOMIZED RANGE FINDER %%%%%%%%%%%%%%%%%%%%%%%%

\subsubsection{Randomized Range Finder}
This is the most naive and simplest implementation of Algorithm \ref{alg:basic}.
Given an oversampling parameter $p$, the \textit{Randomized Range Finder}
performs Algorithm \ref{alg:basic} with a gaussian test matrix 
$\mtx{\Omega}\in\Rspace{n\times l}$ with $l=k+p$ and $k$ being a pre-specified
target rank. Then, it orthogonalizes the rows of the resulting matrix $\mtx{Y}$ by computing
a QR decomposition. A numerical issue arises when computing the orthogonalization
procedure due to the fact that the columns of $\mtx{Y}$ are almost linearly
dependent. The authors in ~\cite{halko2011finding} found that
using the \textit{double orthogonalization} 
~\cite{bjorck1994numerics} was enough to guarantee stability
of the procedure.

\begin{figure}[ht]
\begin{center}
\fbox{
\begin{minipage}{.9\textwidth}
\begin{center}
\textsc{Randomized Range Finder}
\end{center}
\begin{tabbing}
\hspace{5mm} \= \hspace{5mm} \= \hspace{5mm} \= \hspace{5mm} \= \kill
\anum{1} \>Draw an $n\times \ell$ Gaussian random matrix $\mtx{\Omega}$.\\
\anum{2} \>Form the $m\times \ell$ matrix $\mtx{Y} = \mtx{A}\mtx{\Omega}$.\\
\anum{3} \>Construct an $m \times \ell$ matrix $\mtx{Q}$ whose columns form an orthonormal\\
         \> basis for the range of $\mtx{Y}$, e.g., using the QR factorization $\mtx{Y} = \mtx{Q}\mtx{R}$.
\end{tabbing}
\end{minipage}}
\end{center}
\end{figure}

The complexity analysis gives

\begin{equation}\label{eq:analysis-rand-finder}
T_{\text{Randomized Range Finder}}\sim \mathcal{O}(mnl)
\end{equation}

this is because generating a gaussian random number is $\mathcal{O}(1)$
and computing a matrix vector multiplication is $\mathcal{O}(mn)$.

%%%%%%%%%%%%%%%%%% ADAPTIVE RANDOMIZED RANGE FINDER %%%%%%%%%%%%%%%%%%%%%%%%

\subsubsection{Adaptive Randomized Range Finder}

One important pitfall of the \textit{Randomized Range Finder} is that it
requires to know in advance the target rank $k$. However, if we intend
to solve the \textit{fixed-precision problem}, we need a scheme to estimate
the error $\|\mtx{A}-\mtx{Q}\mtx{Q}^\adj\mtx{A}\|$ during the algorithm
in order to match the requited tolerance $\epsilon$.

This scheme is possible and it is direct consequence of the following lemma.
\begin{lemma}
\label{thm:aposteriori}
Let $\mtx{B}$ be a real $m\times n$ matrix.
Fix a positive integer $r$ and a real number $\alpha > 1$.
Draw an independent family $\{ \vct{\omega}^{(i)} : i = 1, 2, \dots, r \}$
of standard Gaussian vectors.  Then
\begin{equation*}
\norm{\mtx{B}}
    \leq \alpha \sqrt{\frac{2}{\pi}} \max_{i = 1, \dots, r}
    \smnorm{}{\mtx{B}\vct{\omega}^{(i)} }
\end{equation*}
except with probability $\alpha^{-r}$.
\end{lemma}

Lemma \ref{thm:aposteriori} says that we can bound the error with high
probability using inexpensive computations in an online manner. 
The Lemma \ref{thm:aposteriori} applied to our problem reads

\begin{equation}
\label{eq:errorest}
\norm{ (\Id - \mtx{Q}\mtx{Q}^{\adj})\mtx{A}}
    \leq 10 \sqrt{\frac{2}{\pi}} \max_{i = 1, \dots, r}
    \smnorm{}{ (\Id - \mtx{Q}\mtx{Q}^{\adj}) \mtx{A}\vct{\omega}^{(i)} }
\end{equation}
with probability at least $1 - 10^{-r}$.

The high probability bound \ref{eq:errorest} gives a simple online scheme
to decide when we have a good enough $\mtx{Q}$ that matches the pre-specified
tolerance. The goal here is to find an integer $l$ and a $m\times l$
orthonormal matrix $\mtx{Q}^{(l)}$ such that
\begin{equation} \label{eqn:err_est_err_bd}
\smnorm{}{ \big(\Id - \mtx{Q}^{(\ell)} (\mtx{Q}^{(\ell)})^{\adj} \big)\mtx{A} } \leq \eps.
\end{equation}

We call \textit{Adaptive Randomized Range Finder} to the
algorithm derived from Lemma \ref{thm:aposteriori} that solves this
problem.

\begin{figure}[ht]
\begin{center}
\fbox{
\begin{minipage}{.9\textwidth}
\begin{center}
\textsc{Adaptive Randomized Range Finder}
\end{center}
\begin{tabbing}
\hspace{5mm} \= \hspace{5mm} \= \hspace{5mm} \= \hspace{5mm} \= \kill
\anum   {1} \> Draw standard Gaussian vectors $\vct{\omega}^{(1)}, \dots, \vct{\omega}^{(r)}$ of length $n$.\\
\anum{2} \> For $i = 1,2,\dots,r$, compute $\vct{y}^{(i)} = \mtx{A}\vct{\omega}^{(i)}$.\\
\anum{3} \> $j=0$.\\
\anum{4} \> $\mtx{Q}^{(0)} = [\ ]$, the $m\times 0$ empty matrix. \\
\anum{5} \> \textbf{while} $\displaystyle
         \max\left\{\smnorm{}{\vct{y}^{(j+1)}},\smnorm{}{\vct{y}^{(j+2)}},\dots,\smnorm{}{\vct{y}^{(j+r)}} \right\} >
\varepsilon/(10\sqrt{2/\pi})$,\\
\anum{6} \> \> $j = j + 1$.\\
\anum{7} \> \> Overwrite $\vct{y}^{(j)}$ by $\bigl(\Id - \mtx{Q}^{(j-1)}(\mtx{Q}^{(j-1)})^{\adj}\bigr)\vct{y}^{(j)}$.\\
\anum{8} \> \> $\vct{q}^{(j)} = \vct{y}^{(j)}/\norm{\vct{y}^{(j)}}$.\\
\anum{9} \> \> $\mtx{Q}^{(j)} = [\mtx{Q}^{(j-1)}\ \vct{q}^{(j)}]$.\\
\anum{10} \> \> Draw a standard Gaussian vector $\vct{\omega}^{(j+r)}$ of length $n$.\\
\anum{11} \> \> $\vct{y}^{(j+r)} = \left(\Id - \mtx{Q}^{(j)}(\mtx{Q}^{(j)})^{\adj}\right)\mtx{A}\vct{\omega}^{(j+r)}$.\\
\anum{12} \> \> \textbf{for} $i = (j+1),(j+2),\dots,(j+r-1)$,\\
\anum{13} \> \> \> Overwrite $\vct{y}^{(i)}$ by $\vct{y}^{(i)} - \vct{q}^{(j)}\ip{\vct{q}^{(j)}}{\vct{y}^{(i)}}$.\\
\anum{14} \> \> \textbf{end for}\\
\anum{15} \> \textbf{end while}\\
\anum{16} \> $\mtx{Q} = \mtx{Q}^{(j)}$.
\end{tabbing}
\end{minipage}}
\end{center}
\end{figure}

One important question is how good the bound given by Lemma \ref{thm:aposteriori}
is in practice. If there is a significant gap between theory and practice
the optimal $l$ will be overestimated. This question will be addressed in the experimental section
REF.

%%%%%%%%%%%%%%%%%% RANDOMIZED POWER ITERATION %%%%%%%%%%%%%%%%%%%%%%%%

\subsubsection{Randomized Power Iteration}

The \textit{Randomized Range Finder} algorithm assumes that the singular values
of the matrix decay fast. This can be seen from equation \ref{eq:iter-range},
where the small singular
values interfere with the calculation of the range. This intuition is made precise 
in THEOREM, where the error of the approximation depends on
the $k+1$-th singular value.

The goal here is to reduce the weight of the
small singular values by taking powers of the matrix whose range we want
to approximate. Instead of applying the sampling scheme to $\mtx{A}$, we will
apply it to $\mtx{B}=(\mtx{A}\mtx{A}^\adj)^q\mtx{A}$ where $q>0$ is a small
integer.

The matrix $\mtx{B}$ has the same singular vectors than $\mtx{A}$ (hence, 
the same range), but its singular values decay much more quickly.

\begin{equation}\label{eq:sing-values-power}
\sigma_j(\mtx{B}) = \sigma_j(\mtx{A})^{2q+1},
\hspace{0.5cm} j=1,2,3,\ldots
\end{equation}

The \textit{Randomized Power Iteration} algorithm is the same 
as the \textit{Randomized Range Finder} but replacing the formula
$\mtx{Y}=\mtx{A}\mtx{\Omega}$ by $\mtx{Y}=\mtx{B}\mtx{\Omega}$.

\begin{figure}[ht]
\begin{center}
\fbox{
\begin{minipage}{.9\textwidth}
\begin{center}
\textsc{Randomized Power Iteration}
\end{center}
\begin{tabbing}
\hspace{5mm} \= \hspace{5mm} \= \hspace{5mm} \= \hspace{5mm} \= \kill
\anum{1} \>Draw an $n\times \ell$ Gaussian random matrix $\mtx{\Omega}$.\\
\anum{2} \>Form the $m\times \ell$ matrix $\mtx{Y} = (\mtx{A}\mtx{A}^{\adj})^{q}\mtx{A}\mtx{\Omega}$ via alternating application\\
         \>of $\mtx{A}$ and $\mtx{A}^{\adj}$.\\
\anum{3} \>Construct an $m \times \ell$ matrix $\mtx{Q}$ whose columns form an orthonormal\\
         \> basis for the range of $\mtx{Y}$, e.g., via the QR factorization $\mtx{Y} = \mtx{Q}\mtx{R}$.
\end{tabbing}
\lsp
{\bf Note:} This procedure is vulnerable to round-off errors; see Remark \ref{remark:roundoff_in_powerscheme}.
The recommended implementation appears as Algorithm~\ref{alg:subspaceiteration}.
\end{minipage}}
\end{center}
\end{figure}

The computational complexity of the algorithm is essentially the same because
it only requires $2q+1$ as many matrix-multiplications as ALGORITHM
but the number $q$ is in practice 2,3 or 4. This can be seen from 
THEOREM, which shows that the power iteration drives the approximation
gap to 1 exponentially fast.


%%%%%%%%%%%%%%%%%% FAST RANDOMIZED RANGE FINDER %%%%%%%%%%%%%%%%%%%%%%%%

\subsubsection{Fast Randomized Range Finder}

A simple inspection to equation \ref{eq:cost_basic} reveals the computational
bottleneck of the sampling procedure. This is the matrix multiplication
$\mtx{Y}=\mtx{A}\mtx{\Omega}$ that takes $\mathcal{O}(mnl)$ operations
for dense matrices, which is the same as the $l$-SVD (computed
after a prior rank-revealing QR factorization
~\cite{gu1996efficient}).

The key idea is to use a \textit{structured} random matrix that allows us 
to compute the product in $\mathcal{O}(mn\log(l))$ operations.

The simplest structured random matrix that meets our goals is the so-called
\textit{subsampled random Fourier transform} (SRFT).

An SRFT is an $n \times \ell$ matrix of the form
\begin{equation}
\label{eq:def_srft}
\mtx{\Omega} = \sqrt{\frac{n}{\ell}} \, \mtx{DFR},
\end{equation}
where
\lsp
\begin{itemize}
\item   $\mtx{D}$ is an $n \times n$ diagonal matrix whose entries are
independent random variables uniformly distributed on the complex unit circle,

\item   $\mtx{F}$ is the $n \times n$ unitary discrete Fourier transform (DFT),
whose entries take the values $f_{pq} = n^{-1/2} \, \econst^{-2\pi\iunit (p-1)(q-1)/n}$ for $p, q = 1, 2, \dots, n$, and

\item   $\mtx{R}$ is an $n \times \ell$ matrix that samples $\ell$ coordinates
from $n$ uniformly at random, i.e., its $\ell$ columns are drawn randomly
without replacement from the columns of the $n \times n$ identity matrix.
\end{itemize}
\lsp

Now, via a subsampled FFT ~\cite{woolfe2008fast}, we can compute the
sample matrix $\mtx{Y}=\mtx{A}\mtx{\Omega}$ with
$\mathcal{O}(mn\log(l))$ operations.

Now, the total number of operations required by this procedure is reduced to
\begin{equation}
\label{eq:cost_SRFT}
T_{\rm struct} \sim mn \log(\ell) + \ell^2 n
\end{equation}


\begin{figure}[ht]
\begin{center}
\fbox{
\begin{minipage}{.9\textwidth}
\begin{center}
\textsc{Fast Randomized Range Finder}
\end{center}
\begin{tabbing}
\hspace{5mm} \= \hspace{5mm} \= \hspace{5mm} \= \hspace{5mm} \= \kill
\anum{1} \>Draw an $n\times \ell$ SRFT test matrix $\mtx{\Omega}$, as defined by \eqref{eq:def_srft}. \\
%\pgnotate{Added reference.}\\
\anum{2} \>Form the $m\times \ell$ matrix $\mtx{Y} = \mtx{A}\mtx{\Omega}$ using a (subsampled) FFT.\\
\anum{3} \>Construct an $m \times \ell$ matrix $\mtx{Q}$ whose columns form an orthonormal\\
         \> basis for the range of $\mtx{Y}$, e.g., using the QR factorization $\mtx{Y} = \mtx{Q}\mtx{R}$.
\end{tabbing}
\end{minipage}}
\end{center}
\end{figure}


\section{Stage 2}
\subsubsection{Compute standard QR and SVD from a general factorization}
Now we will specify how we can compute the standards SVD and QR decompositions
from a general low rank decomposition $\|\mtx{A} - \mtx{B}\mtx{C}\|\leq\epsilon$.

\begin{itemize}
  \item \textit{SVD decomposition:}
\lsp
\begin{enumerate}
\item Compute a QR factorization of $\mtx{B}$ so that $\mtx{B} = \mtx{Q}_{1}\mtx{R}_{1}$.
\item Form the product $\mtx{D} = \mtx{R}_{1}\mtx{C}$, and compute an SVD:
      $\mtx{D} = \mtx{U}_{2}\mtx{\Sigma}\mtx{V}^{\adj}$.
\item Form the product $\mtx{U} = \mtx{Q}_{1}\mtx{U}_{2}$.
\end{enumerate}
\lsp
The result is a diagonal matrix $\mtx{\Sigma}$ and orthonormal matrices $\mtx{U}$ and $\mtx{V}$
such that $\norm{\mtx{A} - \mtx{U}\mtx{\Sigma}\mtx{V}^{\adj}} \leq \varepsilon$.

\item \textit{QR decomposition:}
\lsp
\begin{enumerate}
\item Compute a QR factorization of $\mtx{B}$ so that $\mtx{B} = \mtx{Q}_{1}\mtx{R}_{1}$.
\item Form the product $\mtx{D} = \mtx{R}_{1}\mtx{C}$, and compute a QR factorization: $\mtx{D} = \mtx{Q}_{2}\mtx{R}$.
\item Form the product $\mtx{Q} = \mtx{Q}_{1}\mtx{Q}_{2}$.
\end{enumerate}
\lsp
The result is an orthonormal matrix  $\mtx{Q}$ and a weakly upper-triangular matrix $\mtx{R}$ such
that $\norm{\mtx{A} - \mtx{Q}\mtx{R}} \leq \varepsilon$.

\end{itemize}

\subsubsection{Factorizations based on forming $\mtx{Q}^\adj\mtx{A}$ directly}
\subsubsection{Other Procedures}

\section{Full algorithms}
