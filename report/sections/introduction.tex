\section{Introduction}

Matrix factorization is listed as one of the most influential set
of techniques during the 20th century ~\cite{dongarra2000guest}, among
the Fast Fourier Transform, MCMC sampling methods and others.
As Stewart ~\cite{stewart2000decompositional} argues, the principle of the
decompositional approach aims to construct computational platforms from
which a variety of problems can be solved.

Although the decompositional approach to matrix computation remains
fundamental, nowadays in the era of big data, most of the classical algorithms
are inadequate to tackle most of the problems.

Data matrices are now incredibly
big, making most of the classical approaches too expensive in terms
of computation. Moreover, it is common in information sciences to have
data which is missing or innaccurate. This gives the opportunity to sacrify
some accuracy on the algorithm to gain on computation, which classical 
algorithms are not able to do. Another important aspect is the role
of data transfer in the computational cost of a given algorithm, i.e, 
techniques that may perform fewer passes over the data may be substantially
faster in practice. 

In this review, we present, analyse and test \textit{randomized} algorithms
for matrix factorizations. This set of novel techniques addresses the issues
stated above, i.e, can trade-off computation and accuracy to an arbitrary
precision, gain on robustness, and reduce the number of passes on big datasets
when data transfer is expensive.

The purpose of approximated low-rank matrix factorization is to factorize
a given matrix $\mtx{A}\in\Rspace{m\times n}$ into a product of
two smaller matrices $\mtx{B}\in\Rspace{m\times k}$ and
$\mtx{C}\in\Rspace{k\times n}$. 

\begin{equation}
\label{eq:lowrank}
\begin{array}{ccccccccccc}
\mtx{A} &\approx& \mtx{B} & \mtx{C},\\
m\times n && m \times k & k\times n.
\end{array}
\end{equation}

The matrix $\mtx{B}\times\mtx{C}$ in \ref{eq:lowrank} is called a rank-$k$ approximation of the
matrix $\mtx{A}$.

The inner dimension $k$ is called the \textit{numerical rank}
of the matrix. This
quantity differs from the \textit{algebraic rank}, which is defined as the 
dimension of the image. The numerical rank is commonly defined as follows

\begin{equation}\label{eq:num-rank}
r(\mtx{A})\defeq\frac{\|\mtx{A}\|_F^2}{\|\mtx{A}\|^2}
= \sum_{j=1}^{\min(m,n)}\left(\frac{\sigma_j}{\sigma_1}\right)^2
\end{equation}

and it gives a better understanding of how accurate a rank-$k$ approximation
can be. Note that we always have $r(\mtx{A})\leq\text{rank}(\mtx{A})$. The notion
of numerical rank appears in ~\cite{vershynin2016high} and has been studied at the Theory Reading Group
course of the master in depth
\footnote{The comparison between both notions of rank can be better understood
through the following caracterization. 
$\text{rank}(\mtx{A})=\dim(\textbf{A}B_2^n)$ and $r(\mtx{A})=d(\mtx{A}B_2^n)$
where $B_2^n$ is the euclidean ball, hence, $\mtx{A}B_2^n$ is the ellipsoid with
the axis of magnitude the singular values $\sigma_j$'s.
Here, $\dim(\cdot)$ denotes the algebraic dimension and
$d(\cdot)$ denotes the \textit{statistical dimension}.
The statistical dimension is defined as $d(T)\defeq \frac{h(T-T)^2}{\text{diam}(T)^2}
\sim \frac{w(T)^2}{\text{diam}(T)^2}$ where $h(T)^2=\Expect\sup_{t\in T}\la g, t\ra^2$
and $w(T)=\Expect\sup_{t\in T}\la g, t\ra$ is the gaussian width.
As discussed in ~\cite{vershynin2016high}, the statistical dimension is a more stable notion
of dimension, in the same way that the numerical rank is more stable than the
algebraic rank.
}.

The task of computing a low-rank approximation to a given matrix can be split into
two computational stages. The first is to construct a low dimensional subspace that
can capture the action of the matrix. The second is to restrict the matrix to the
low dimensional subspace and then compute a standard factorization (QR, SVD, etc)
of the reduced matrix.

\begin{itemize}
  \item\textbf{Stage 1:} Compute an approximate basis for the range of the input
  matrix $\mtx{A}$. We want to find a matrix $\mtx{Q}$ with a small
  number of orthonormal columns such that

  \begin{equation}\label{eq:app-basis}
    \mtx{A}\approx \mtx{Q}\mtx{Q}^\adj\mtx{A}
  \end{equation}

  The main idea is to approximate the range of the matrix via a randomized
  method. This can be accomplised by iteratively computing the image of
  random vectors from the input space and then orthogonalizing.
  All the randomness of these methods will belong to this first stage.

  \item\textbf{Stage 2:} Given a matrix $\mtx{Q}$ that satisfies \ref{eq:app-basis},
  compute a standard factorization (QR, SVD, etc.) of $\mtx{A}$.  Note that
  taking $\mtx{B}=\mtx{Q}$ and $\mtx{C}=\mtx{Q}^\adj\mtx{A}$ we already have
  a low rank approximation of the matrix.
   There is no randomness at this stage and only classical linear algebra computations are
  incolved.
\end{itemize}

During the rest of the introduction, we will review some basics about matrix approximation
and we will provide to the reader the basic aspects and insights of both stages, 
which will be further studied in depth in the main body.


\subsection{Approximating the range of a matrix via randomness}

The problem of finding the best $\epsilon$-approximation of a given matrix $\mtx{A}$
 is called the \textit{fixed-precision approximation problem}. More concretely, we are given
 a tolerance $\epsilon$ and the goal is to find a matrix $\mtx{Q}$
 with $k=k(\epsilon)$ columns such that

\begin{equation}\label{eq:fixed-precision}
\|\mtx{A} - \mtx{Q}\mtx{Q}^\adj\mtx{A}\|\leq\epsilon
\end{equation}

The goal here is to find a $\mtx{Q}$ with the smaller number of columns
possible.

Another closely related problem is the so-called \textit{fixed-rank approximation problem}, 
which seeks to find the best rank-$k$ approximation of the matrix.

\begin{equation}
\label{eq:fixed-rank}
\min_{{\rm rank}(\mtx{X}) \leq j}\norm{ \mtx{A} - \mtx{X} }.
\end{equation}

The Singular Value Decomposition (SVD)
\footnote{The Singular Value Decomposition is a central tool for a data scientist.
It is the algorithmic tool used to perform \textit{Principal Component Analysis} (PCA) on a dataset
$\mtx{X}\in\Rspace{N\times d}$ of size $N$ and dimensionality $d$. PCA finds
linear hidden linear structure in the data by looking at the eigenvectors with largest
eigenvalues of the empirical covariance matrix of the data
$\hat{\mtx{\Sigma}}_N = \mtx{X}^\transp\mtx{X}=\frac{1}{N}\sum_{j=1}^N\mtx{x}_j\mtx{x}_j^\transp$.
These eigenvectors and eigenvalues can also be found by computing the SVD of the data matrix
$\mtx{X}=\mtx{U}\mtx{D}\mtx{V}^\transp$, because $\hat{\mtx{\Sigma}}_N = \mtx{V}\mtx{D}^2\mtx{V}^\transp$.
One important remark is that PCA does not suffer from the curse of dimensionality. For example,
if we assume sub-gaussianity on the random variable $X$, then we have
$\Expect\| \hat{\mtx{\Sigma}}_N - \mtx{\Sigma} \|\leq \varepsilon \|\mtx{\Sigma}\|$
if we take a sample size of $N\sim\varepsilon^{-2}d$, i.e, the number of samples
scales linearly with the dimensionality of the dataset. If instead of sub-gaussianity,
we relax it to $\|X\|_2\leq K(\Expect\|X\|_2^2)^{1/2}$ we only get an extra log-term.
This is one of the main reasons why PCA works!}
is key to analyze this problem. Recall that
the SVD of a matrix $\mtx{A}$ is the following decomposition

\begin{equation}\label{eq:svd}
\mtx{A} = \mtx{U}\mtx{\Sigma}\mtx{V}^\adj
= \sum_{j=1}^{\text{rank}(\mtx{A})}\sigma_j\mtx{u}_j\mtx{v}_j^\adj
\end{equation}

where $\{\mtx{u}_k\}_k, \{\mtx{v}_k\}_k$ are orthonormal basis on the output and input
space respectively, $\sigma_1\geq\sigma_2,\ldots,\sigma_{\text{rank}(\mtx{A})}\geq 0$
are the ordered singular values.

 The SVD provides an optimal answer to the
 \textit{fixed-rank approximation problem} ~\cite{mirsky1960symmetric} through the following
 important observation

\begin{equation}
\label{eqn:mirsky}
\min_{{\rm rank}(\mtx{X}) \leq k}\norm{ \mtx{A} - \mtx{X} } = \sigma_{k+1}.
\end{equation}

It is straightforward to check that the optimum is attained at
 $\mtx{X}_{*} = \sum_{j=1}^k\sigma_j\mtx{u}_j\mtx{v}_j^\adj$, namely, the $k$-truncated
 SVD (k-SVD) of the matrix $\mtx{A}$. More precisely, we have that
$\mtx{B}=\mtx{U}\mtx{\Sigma}_{[k]}^{1/2}$ and $\mtx{C} = \mtx{\Sigma}_{[k]}^{1/2}\mtx{V}^\adj$
are the best solutions \ref{eq:lowrank} when the rank is fixed.

Let's suppose now that we know the desired rank $k$ in advance. The goal is 
to find a matrix $\mtx{Q}$ with $k+p$ orthonormal columns such that

\begin{equation}\label{eq:fixed-rank-app}
\|\mtx{A}-\mtx{Q}\mtx{Q}^\adj\mtx{A}\| \approx
\min_{{\rm rank}(\mtx{X}) \leq k}\norm{ \mtx{A} - \mtx{X} }
\end{equation}

where $p$ is called the \textit{oversampling parameter}.

% EXPLICACIO

\subsection{Intuition of the randomized method to find Q} \label{alg:proto-algorithm}

The key observation that leverage these methods is the fact that the matrix
$\mtx{Q}$ can be found by sampling, and now the reader will obtain intuition
on how this can be done.

Suppose we seek a basis for the range of a matrix $\mtx{A}$ with algebraic rank
$k$. Random elements $\mtx{y}^{(i)}$ from the range can be computed by computing the image
of random vectors $\mtx{w}^{(i)}$ from the input space. Let us repeat this
process $k$ times:

\begin{equation}\label{eq:iter-range}
\mtx{y}^{(i)} = \mtx{A}\mtx{w}^{(i)}, \hspace{0.5cm}i=1,\ldots,k
\end{equation} 

Thanks to the randomness, the set $\{\mtx{w}^{(i)}\}_{i=1}^k$ is likely to be
in general linear position and no vector will fall in $\ker\mtx{A}$
if this is a set of measure zero under the probability measure we sampled from.
Therefore, an orthogonalization procedure gives the desired orthonormal basis.

What happens if the matrix $\mtx{A}$ has not exact algebraic rank equal to $k$
\footnote{In practice this is always the case.}?
Write $\mtx{A} = \mtx{B} + \mtx{E}$ where $\mtx{B}$ is a rank-$k$ matrix containing
the information we seek and $\mtx{E}$ a small perturbation.
We want a basis that covers the range of $\mtx{B}$, however, if we repeat
procedure \ref{eq:iter-range}, the vectors will be affected by the perturbation
and the $\{\mtx{y}^{(i)}\}_{i=1}^k$ will have small components that will make
them fall outside the desired space.

To overcome this issue, the idea is to take $p$ more samples:

\begin{equation}\label{eq:iter-range}
\mtx{y}^{(i)} = \mtx{A}\mtx{w}^{(i)} = \mtx{B}\mtx{w}^{(i)}
+ \mtx{E}\mtx{w}^{(i)}, \hspace{0.5cm}i=1,\ldots,k+p
\end{equation}

The enriched set $\{\mtx{y}^{(i)}\}_{i=1}^{k+p}$ has much more chance
of spanning the desired subspace, and this is grounded with some
theoretical results; see for example Theorem \ref{thm:avg-frob-error-gauss}.
The theory also shows that
$p$ can be quite small. In practice, $p=5$ is enough.

Here in the box below, we illustrate the basic prodecure to
find $\mtx{Q}$ that approximates the range of $\mtx{A}$ through randomization.
We call it Proto-Algorithm.
\begin{figure}[ht]
\begin{center}
\framebox{\begin{minipage}{.9\textwidth}
\begin{center}
\textsc{Proto-Algorithm: Solving the Fixed-Rank Problem}
\end{center}
\begin{tabbing}
\hspace{5mm} \= \hspace{5mm} \= \hspace{5mm} \= \hspace{5mm} \= \kill
\anum{1} \>Draw a random $n \times (k + p)$ test matrix $\mtx{\Omega}$.\\
\anum{2} \>Form the matrix product $\mtx{Y} = \mtx{A\Omega}$.\\
\anum{3} \>Construct a matrix $\mtx{Q}$ whose columns form an orthonormal basis for \\
         \>the range of $\mtx{Y}$.
\end{tabbing}
\end{minipage}}
\end{center}
% \caption{}
\end{figure}

\subsection{Construction of standard matrix factorizations from Q}

This corresponds to \textbf{Stage 2} of the algorithm. Once we have $\mtx{Q}$
such that $\mtx{A}\approx\mtx{Q}\mtx{Q}^\adj\mtx{A}$, taking 
$\mtx{B}=\mtx{Q}$ and $\mtx{C}=\mtx{Q}^\adj\mtx{A}$ we directly produce a low rank
matrix decomposition $\mtx{A}\approx\mtx{B}\mtx{C}$. However, 
there are more involved methods to produce a decomposition from $\mtx{Q}$ with
computational advantadges compared to the direct method. 

One we have $\mtx{A}\approx\mtx{B}\mtx{C}$, standard matrix decompositions 
(e.g. SVD, QR) can be easily derived as we show in \ref{itm:SVD-from-C}. \\



A lot of questions need to be addressed in order to turn these methods into
a technology or \textit{off-the-shelf} algorithms. The following 
sections will we devoted to this.
